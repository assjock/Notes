\section{Symmetric, alternating, and dihedral groups}
\begin{ex}
    Find four different subgroups of $S_{4}$ that are isomorphic to $S_{3}$ and nine isomorphic to $S_{2}$.
\end{ex}

\begin{answer}
    $S_{4}=\{(1), (12), (13), (14), (23), (24), (34), (123), (124), (132),\\ (142), (134), (143), (234), (243), (12)(34), (13)(24), (14)(23), (1234),\\ (1243), (1324), (1342), (1423), (1432)\}$.

    $A_{1}=\{(1), (12), (13), (23), (123), (132)\}$;

    $A_{2}=\{(1), (12), (14), (24), (124), (142)\}$;

    $A_{3}=\{(1), (13), (14), (34), (134), (143)\}$;

    $A_{4}=\{(1), (23), (24), (34), (234), (243)\}$;
    
    $A_{1}\cong A_{2}\cong A_{3}\cong A_{4}$.

    $B_{1}=\{(1), (12)\}$; $B_{2}=\{(1),(13)\}$; $B_{3}=\{(1),(14)\}$; $B_{4}=\{(1), (23)\}$; $B_{5}=\{(1),(24)\}$; $B_{6}=\{(1), (34)\}$; $B_{7}=\{(1),(12)(34)\}$; $B_{8}=\{(1),(13)(24)\}$; $B_{9}=\{(14)(23)\}$;

    $B_{1}\cong B_{2}\cong B_{3}\cong B_{4}\cong B_{5}\cong B_{6}\cong B_{7}\cong B_{8}\cong B_{9}$.
\end{answer}

$$ $$

\begin{ex}
    \begin{enumerate}[(a)]
        \item $S_{n}$ is generated by the $n-1$ transpositions $(12)$, $(13)$, $(14)$, $\dots$, $(1n)$.
        \item $S_{n}$ is generated by the $n-1$ transpositions $(12), (23), (34),\dots, (n-1\, n)$.
    \end{enumerate}
\end{ex}

\begin{answer}
    \begin{enumerate}[(a)]
        \item $\forall x\in S_{n}$, $x$ can be written as a product of transpositions. Actually, for any transposition $(ij)$, we can obtain it by $(1i)(1j)(1i)=(ij)$. So $x\in \left\langle (12), (13),\dots,(1n)\right\rangle$, $S_{n}\subset\left\langle(12), (13),\dots,(1n)\right\rangle$.
        \item We can contruct $(1i)$ inductively since $(1i)=(1\, i-1)(i-1\, i)(1\, i-1)$. From (a), we have $\forall x\in S_{n}$, $x\in \left\langle (12), (13),\dots,(1n)\right\rangle$. Thus $S_{n}\subset\left\langle(12), (13),\dots,(1n)\right\rangle\subset \left\langle (12), (23), (34),\dots, (n-1\, n)\right\rangle$.
    \end{enumerate}
\end{answer}

$$ $$

\begin{ex}
    If $\sigma=(i_{1}i_{2}\cdots i_{r})\in S_{n}$ and $\tau\in S_{n}$, then $\tau\sigma\tau^{-1}$ is the $r$-cycle $(\tau(i_{1})\tau(i_{2})\cdots\tau(i_{r}))$.
\end{ex}

\begin{answer}
    $\sigma(i_{n})=i_{n+1}$ for $n=1,2,\dots,r-1$, $\sigma(i_{r})=i_{1}$. Assume $\tau(i_{n})=j_{n}$, $n=1,2,\dots,r-1$ and $I=\{i_{n}|n=1,2,\dots,r-1\}$, $J=\{j_{n}|n=1,2,\dots,r-1\}$. For $x\notin J$, $\tau\sigma\tau^{-1}(x)=\tau\tau^{-1}(x)=x$. For $x=j_{k}\in J$, $\tau^{-1}(x)=i_{k}$, $\sigma(\tau^{-1}(x))=i_{k+1}$, $\tau(\sigma(\tau^{-1}(x)))=j_{k+1}$ and $\tau\sigma\tau^{-1}(j_{r})=j_{1}$. Thus $\tau\sigma\tau^{-1}=(\tau(i_{1})\tau(i_{2})\cdots\tau(i_{r}))$.
\end{answer}

$$ $$

\begin{ex}
    \begin{enumerate}[(a)]
    \item $S_{n}$  is generated by $\sigma_{1}=(12)$ and $\tau=(123\cdots n)$.
    \item $S_{n}$ is generated by $(12)$ and $(23\cdots n)$.
    \end{enumerate}
\end{ex}

\begin{answer}
    \begin{enumerate}[(a)]
        \item Denote $\sigma_{i}=\tau\sigma_{i-1}\tau^{-1}$. Applying \textbf{Exercise 1.6.3}, $\sigma_{i}=(i\, i+1)$. By \textbf{Exercise 1.6.2}(b), $S_{n}\subset\left\langle (12), (23), (34),\dots, (n-1\, n)\right\rangle=\left\langle\sigma_{1}, \sigma_{2},\dots,\sigma_{n-1}\right\rangle\subset \left\langle \tau, \sigma_{1}\right\rangle$. $S_{n}$ can be generated by $\tau$ and $\sigma_{1}$.
        \item Denote $\sigma_{1}=(12)$, $\tau=(23\cdots n)$, $\sigma_{i}=\tau\sigma_{i-1}\tau^{-1}$. Applying \textbf{Exercise 1.6.3},$\sigma_{i}=(1\, i+1)$. By \textbf{Exercise 1.6.2}(a), $S_{n}\subset\left\langle(12), (13),\dots,(1n)\right\rangle=\left\langle \sigma_{1},\sigma_{2},\dots,\sigma_{n-1}\right\rangle\subset\left\langle \tau,\sigma_{1}\right\rangle$. $S_{n}$ can be generated by $\tau$ and $\sigma_{1}$.
    \end{enumerate}
\end{answer}

$$ $$

\begin{ex}
    Let $\sigma, \tau \in S_{n}$. If $\sigma$ is even (odd), then so is $\tau\sigma\tau^{-1}$.
\end{ex}

\begin{answer}
    Assume $\sigma=(x_{1}x_{2})\cdots(x_{2n-1}x_{2n})$, $\tau=(y_{1}y_{2})\cdots(y_{2m-1}y_{2m})$. Then $\tau^{-1}=(y_{2m-1}y_{2m})\cdots(y_{1}y_{2})$. $\sigma$ is odd (even) if an only if $n$ is odd (even). $\tau\sigma\tau^{-1}$ has $2m+n$ transpositions. We can add $(ij)=(ji)=(1)$ into some segments of $\tau\sigma\tau^{-1}$ without changing it. So $\tau\sigma\tau^{-1}$ is odd (even) if and only if $2m+n$ is odd (even). $2m+n\equiv n\mod 2$ so $\tau\sigma\tau^{-1}$ is odd (even) if and only if $\sigma$ is odd (even).
\end{answer}

$$ $$

\begin{ex}
    $A_{n}$ is the only subgroup of $S_{n}$ of index 2.
\end{ex}

\begin{answer}
    For any subgroup $N<S_{n}$ and $\left[S_{n}:N\right]=2$, we have $N\lhd S_{n}$.

    Assume there exists $k$-circle $\sigma=(i_{1}i_{2}\cdots i_{k})\in N$. Then for any other $k$-circle $(j_{1}j_{2}\cdots j_{k})$, take $\tau=(i_{i}j_{1})(i_{2}j_{2})\cdots(i_{k}j_{k})$, by \textbf{Exercise 1.6.3}, $\tau\sigma\tau^{-1}=(j_{1}j_{2}\cdots j_{k})\in N$. Thus $N$ contains all the $k$-circles.

    For $n\geq 5$. If there exists 3-circle in $N$, then all the 3-circles are contained in $N$, $A_{n}\subset N\subset S_{n}\Rightarrow A_{n}=N$.

    If there exists 2-circle in $N$, then all the 2-circles are contained in $N$. Notice $(1i)(1j)=(1ij)\in N$ is a 3-circle, so $A_{n}=N$.

    If there only contain $x$ in the form of $(a_{i}a_{2}\cdots a_{n_{1}})(b_{1}b_{2}\cdots b_{n_{2}})\cdots$ where $n_{i}\geq 4$ and every two circles are disjoint. Take $\tau_{i}:\{a_{i}|i=1,2,\dots,n_{1}\}\to \{a_{i}|i=1,2,\dots,n_{1}\}$. We can obtain product of two $n_{1}$-circles\[x^{-1}\tau x \tau^{-1}=(a_{1}a_{2}\cdots a_{n_{1}})(\tau(a_{1})\tau(a_{2})\cdots \tau(a_{n}))\in N\] By the arbitrariness of $\tau$, take \[(\tau(a_{1})\tau(a_{2})\cdots \tau(a_{n}))=(a_{1}a_{4}a_{5}\cdots a_{n}a_{3}a_{2})\] then $x^{-1}\tau x \tau^{-1}=(a_{1}a_{3})(a_{2}a_{4})$ is a product of 2-circles. We can take $a_{1}, a_{2}, a_{3}, a_{4}$ arbitrarily. WLOG, take $(12)(34)\in N$ and $(12)(35)\in N$, $(12)(35)(12)(34)=(345)\in N$. Then there exists 3-circle in $N$, $N=A_{n}$.

    In conlusion, when $n\geq 5$, $S_{n}$ has only one normal subgroup $A_{n}$.

    For $n=2,3,4$, we can verify it by enumeration.
\end{answer}

$$ $$

\begin{ex}
    Show that $N=\{(1),(12)(34),(13)(24),(14)(23)\}$ is a normal subgroup of $S_{4}$ contained in $A_{4}$ such that $S_{4} /N\cong S_{3}$ and $A_{4} /N\cong Z_{3}$.
\end{ex}

\begin{answer}
    Assume $\sigma=(i_{1}i_{2})(i_{3}i_{4})\in N$, $\forall \tau\in S_{4}$, $\tau(i_{n})=j_{n}$, $J=\{j_{n}|n=1,2,3,4\}$. For $x\notin J$, $\tau\sigma\tau^{-1}(x)=\tau\tau^{-1}(x)=x$. For $x=j_{k}\in J$, $\tau^{-1}(x)=i_{k}$, $\sigma\tau^{-1}(x)=i_{3k-4\left[\frac{k}{2}\right]-1}$, $\tau\sigma\tau^{-1}(x)=(\tau(i_{i})\tau(i_{2}))(\tau(i_{3})\tau(i_{4}))\in N$. So $N\lhd S_{4}$. $S_{4} /N=\{N, N(12), N(13), N(23), N(123), N(132)\}\cong S_{3}$. $A_{4} /N=\{N, N(123), N(132)\}\cong Z_{3}$.
\end{answer}

$$ $$

\begin{ex}
    The group $A_{4}$ has no subgroup of order $6$.
\end{ex}

\begin{answer}
    $\left| A_{4} \right| =12$, assume there exists $N<A_{4}$, $\left| N \right| =6$. Then $N\lhd A_{4}$. From \textbf{Exercise 1.6.6}, we know that all 3-circles are contained in $N$. But there're 8 3-circles in total, so $N$ can't exist.
\end{answer}

$$ $$

\begin{ex}
    For $n\geq 3$ let $G_{n}$ be the multiplicaive group of complex matrices generated by $x=\begin{pmatrix}
        0&1\\1&0
    \end{pmatrix}$ and $y=\begin{pmatrix}
        e^{2\pi i/n}&0\\0&e^{-2\pi i/n}
    \end{pmatrix}$, where $i^{2}=-1$. Show that $G_{n}\cong D_{n}$.
\end{ex}

\begin{answer}
    Take a mapping $f:G_{n}\to D_{n}$ as $f(x)=(2\,n)(3\,n-1)\cdots$, $f(y)=(123\cdots n)$. $\left| f(x) \right|=\left| x \right| =2$, $\left| f(y) \right| =\left| y \right| =n$. $f$ is obviously a monomorphism. $\forall a\in D_{n}$, $a=f(x)^{n}f(y)^{m}, m=1,2$, then $a=f(x^{n}y^{m})$, $f$ is a epimorphism. Thus $G_{n}\cong D_{n}$. 
\end{answer}

$$ $$

\begin{ex}
    Let $a$ be the generator of order $n$ of $D_{n}$. Show that $\left\langle a\right\rangle\lhd D_{n}$ and $D_{n} /\left\langle a\right\rangle\cong Z_{2}$.
\end{ex}

\begin{answer}
    $\left| \left\langle a\right\rangle \right| =n$, $b$ is the other generator of $D_{n}$, $a^{n}=b^{2}=(1)$. $\forall k\in \mathbf{Z}$, $a^{k}b=ba^{-k}$ can be easily proved by induction. So $\forall x=a^{m}b^{n}\in D_{n}$, $x=a^{m'}b^{n'}$, here $m'\equiv m\mod 2$, $n'\equiv n\mod 2$. $D_{n}=\{e, a, a^{2}, \dots, a^{n-1}, b, ba, \\\dots, ba^{n-1}\}$. $\left| D_{n} \right| =2n$. Thus, $\left\langle a\right\rangle\lhd D_{n}$. $D_{n} /\left\langle a\right\rangle=\{\left\langle a\right\rangle, \left\langle a\right\rangle b\}\cong Z_{2}$.
\end{answer}

$$ $$

\begin{ex}
    Find all normal subgroups of $D_{n}$.
\end{ex}

\begin{answer}
    The subgroups of $\left\langle a\right\rangle$ is always normal in $D_{n}$. $\left\langle a^{m}\right\rangle <\left\langle a\right\rangle$. $\forall x\in D_{n}$ and $a^{km}\in \left\langle a^{m}\right\rangle$, $x=a^{t}$ or $x=ba^{t}$. \[x^{-1}a^{km}x=a^{-t}a^{km}a^{t}=a^{km}\in\left\langle a^{m}\right\rangle\] or \[x^{-1}a^{km}x=a^{-t}b^{-1}a^{km}ba^{t}=a^{-t}ba^{km}ba^{t}=a^{-t}a^{-km}b^{2}a^{t}=a^{-km}\in \left\langle a^{m}\right\rangle\] so $\left\langle a^{m}\right\rangle \lhd D_{n}$.

    Consider the subgroup $S$ which only contains $ba^{i}, i=1,\dots, n$. Since $ba^{i}\cdot ba^{j}=a^{j-i}\in S\,(i\neq j)$, so $S=\{e, ba^{k}\}$.

    If $n$ is odd, take $x=a^{\frac{n-1}{2}}\in D_{n}$. \[x^{-1}ba^{k}x=a^{\frac{1-n}{2}}ba^{k}a^{\frac{n-1}{2}}=ba^{k+n-1}\notin S\] so $S\ntriangleleft D_{n}$ for all $k=1,2,\dots,n$.

    If $n$ is even, take $x=a^{\frac{n-2}{2}}\in D_{n}$, $n\geq 6$. \[x^{-1}ba^{k}x=a^{\frac{2-n}{2}}ba^{k}a^{\frac{n-2}{2}}=ba^{k+n-2}\notin S\] so $S\ntriangleleft D_{n}$ for all $k=1,2,\dots,n$.

     If $n=2$, all the subgroups are normal since $\left| D_{2} \right| =4$.

     For subgroup $S$ contains both $ba^{i}$ and $a^{j}$. It can be written as $S=\left\langle a^{d}, ba^{r}\right\rangle$, where $d|n$, $0\leq r\leq d-1$. If $\exists a^{m}, a^{n}\in S$, $(m,n)=d$, then there exist $x,y\in \mathbf{Z}$ s.t. $a^{mx+ny}=a^{d}\in \mathbf{Z}$. Thus, $S=\left\langle a^{d},ba^{r}\right\rangle$.

     Take $x=a^{\frac{n-w}{2}}$, then $x^{-1}ba^{r}x=ba^{r+n-w}$.
     
     If $d\geq 3$, take $w\equiv n\mod 2$, $x^{-1}ba^{r}x\notin S$.

     If $d=2$, then $n=2s$ and $S=\{e, a^{s}, ba^{s}, b\}$. $Sa^{k}=\{a^{k}, a^{s+k},ba^{s-k}, ba^{-k}\}$, $k=1,2,\dots, s-1$. $ba^{k}=ba^{-k}$ or $ba^{k}=ba^{s-k}\Rightarrow k=\frac{s}{2}$. So for $s=2$, $n=4$, $S$ is a normal subgroup of $D_{4}$.
\end{answer}

$$ $$

\begin{ex}
    The center of the group $D_{n}$ is $\left\langle e\right\rangle$ if $n$ is odd and isomorphic to $Z_{2}$ if $n$ is even.
\end{ex}

\begin{answer}
    If $n$ is odd, $C$ is the center of $D_{n}$, $C\lhd D_{n}\Rightarrow C<\left\langle a\right\rangle$. Take $a^{d}\in C$, $x=ba^{m}$, \[x^{-1}ax=a^{-m}b^{-1}a^{d}ba^{m}=a^{-m}ba^{d}ba^{m}=a^{-d}=a^{d}\] so $d=0$, $C=\{e\}$.

    If $n$ is even, $n\geq 6$. $C$ is the center of $D_{n}$. $C\lhd D_{n}\Rightarrow C<\left\langle a\right\rangle$ or $C=\{e,ba^{k}\}$.
    
    If $C=\{e,ba^{k}\}$, $C\cong Z_{2}$.
    
    If $C<\left\langle a\right\rangle$,take $a^{d}\in C$, $x=ba^{m}$, \[x^{-1}ax=a^{-m}b^{-1}a^{d}ba^{m}=a^{-m}ba^{d}ba^{m}=a^{-d}=a^{d}\] so $d=\frac{n}{2}$ or $d=0$, $C=\{a^{\frac{n}{2}},e\}\cong Z_{2}$.
\end{answer}

$$ $$

\begin{ex}
    For each $n\geq 3$ let $P_{n}$ be a regular polygon of $n$ sides (for $n=3$, $P_{n}$ is an equilateral triangle; for $n=4$, a square). A \emph{symmetry} of $P_{n}$ is a bijection $P_{n}\to P_{n}$ that preserves distances and maps adjacent vertices on to adjacent vertices.
    
    \begin{enumerate}[(a)]
        \item The set $D_{n}^{*}$ of all symmetries of $P_{n}$ is a group under the binary operation of composition of functions.
        \item Every $f\in D_{n}^{*}$ is completely determined by its actions on the vertices of $P_{n}$. Number the vertices consecutively $1,2,\dots, n$; then each $f\in D_{n}^{*}$ determines a unique permutation $\sigma_{f}$ of $\{1,2,\dots, n\}$. The assignment $f\mapsto \sigma_{f}$ defines a monomorphism of groups $\varphi:D_{n}^{*}\to S_{n}$.
        \item $D_{n}^{*}$ is generated by $f$ and $g$, where $f$ is a rotation of $2\pi /n$ degrees about the center of $P_{n}$ and $g$ is a reflection about the ``diameter'' through the center and vertex 1.
        \item $\sigma_{f}=(123\cdots n)$ and $\sigma_{g}=\begin{pmatrix}
            1&2&3&\cdots&n-1&n\\1&n&n-1&\cdots&3&2
        \end{pmatrix}$, whence \\$\mathrm{Im}\varphi=D_{n}$ and $D_{n}^{*}\cong D_{n}$.
    \end{enumerate}
\end{ex}

\begin{answer}
    In the following analysis, all the numbers are $\mod n$.
    \begin{enumerate}[(a)]
        \item Consider $n$ points $A_{i}=(\cos \frac{2\pi i}{n},\sin \frac{2\pi i}{n})^{t}$, $i=1,2,\dots,n$. $f$ is the transposition of $A_{i}\mapsto A_{j}$ with the consevation of $n$ regular polygon structure. So $f$ must be a bijection. $D_{n}^{*}$ is the set of $f$. By the definition, $D_{n}^{*}\subset S_{n}$. We prove $D_{n}^{*}$ is ta subgroup of $S_{n}$.
        
        Notice that $A_{i+1}=\begin{pmatrix}
            \cos \frac{2\pi i}{n}&-\sin\frac{2\pi i}{n}\\\sin\frac{2\pi i}{n}&\cos \frac{2\pi i}{n}
        \end{pmatrix}A_{i}$.
        
        Denote $X=\begin{pmatrix}
            \cos \frac{2\pi i}{n}&-\sin\frac{2\pi i}{n}\\\sin\frac{2\pi i}{n}&\cos \frac{2\pi i}{n}
        \end{pmatrix}$. To construct the polygon,  we must have \[f(A_{i+1})=\begin{pmatrix}
            \cos \frac{2\pi i}{n}&-\sin\frac{2\pi i}{n}\\\sin\frac{2\pi i}{n}&\cos \frac{2\pi i}{n}
        \end{pmatrix}f(A_{i})\] or \[f(A_{i})=\begin{pmatrix}
            \cos \frac{2\pi i}{n}&-\sin\frac{2\pi i}{n}\\\sin\frac{2\pi i}{n}&\cos \frac{2\pi i}{n}
        \end{pmatrix}f(A_{i+1})\] We need to verify that $\forall f_{1}, f_{2}\in D_{n}^{*}$, $f_{1}f_{2}^{-1}\in D_{n}^{*}$. Assume $B_{i}=f_{2}(A_{i})$, $B_{i+1}=f_{2}(A_{i+1})$. Then $B_{i}=XB_{i+1}$ or $B_{i}=X^{-1}B_{i+1}$. Denote $B_{i}=A_{j}$, then $B_{i+1}=A_{j-1}$ or $B_{i+1}=A_{j+1}$. WLOG, assume $B_{i+1}=A_{j+1}$, then $f_{1}(A_{j})=Xf_{1}(A_{j+1})$ or $f_{1}(A_{j})=X^{-1}f_{1}(A_{j-1})$. So $f_{1}f_{2}^{-1}\in D_{n}^{*}$. $D_{n}^{*}$ is a subgroup of $S_{n}$.
        \item Assume $A_{i}=f(A_{1})$. If $f(A_{2})=A_{i+1}$, since $f$ is a bijection, by induction, we can prove $f(A_{k})=A_{k+i-1}$. $\varphi: D_{n}^{*}\to S_{n}$ can be defined as $\varphi: f\mapsto (1i\,2i-1\,3i-2\cdots)$. If $f(A_{2})=A_{i-1}$, similarly, we can also prove $f(A_{k})=A_{i+1-k}$. $\varphi$ can be defined as $\varphi: f\mapsto (1i)(2\,i-1)(3\,i-2)\cdots$. This means $f$ is completely determined by $f(A_{1})$ and $f(A_{2})$. $D_{n}^{*}$ can be embedded into $S_{n}$.
        \item Denote $\alpha=\begin{pmatrix}
            \cos \frac{2\pi i}{n}&-\sin\frac{2\pi i}{n}\\\sin\frac{2\pi i}{n}&\cos \frac{2\pi i}{n}
        \end{pmatrix}$, $\beta=\begin{pmatrix}
            1&0\\0&1
        \end{pmatrix}$. $f:A_{i}\mapsto \alpha A_{i}$, $g:A_{i}\mapsto \beta A_{i}$. $f$ is the rotation of $\frac{2\pi}{n}$ degrees counter-clockwisely. $g$ is the reflection about $x$-axis. Now we prove $\forall x\in D_{n}^{*}$, $x$ can be factorised into finite product of $f$ and $g$. From (b), $x$ is fully defined by $x(A_{1})$ and $x(A_{2})$. Assume $x(A_{1})=A_{i}$.

        If $x(A_{2})=A_{i+1}$, $x(A_{k})=A_{i-1+k}=\alpha^{i-1}A_{k}$, $k=1,2,\dots,n$. So $x=f^{i-1}$.

        If $x(A_{2})=A_{i-2}$, $x(A_{k})=A_{i+1-k}=\alpha^{i+1}A_{-k}=\alpha^{i+1}\beta A_{k}$. So $x=f^{i+1}g$. Thus $D_{4}^{*}\subset\left\langle f,g\right\rangle$.
        \item $\alpha^{n}=\beta^{2}=\begin{pmatrix}
            1&0\\0&1
        \end{pmatrix}$. We can easily verify that $\left| f \right| =n$ and $\left| g \right| =2$. From \textbf{Exercise 1.6.9}, $\left\langle f,g\right\rangle\cong D_{n}$, $\left| \left\langle f,g\right\rangle \right| =\left| D_{n} \right| =2n$. From (b), $x\in D_{n}^{*}$ if completely determined by $x(A_{1})$ and $x(A_{2})$. There are $2n$ different ways to obtain $x(A_{1})$ and $x(A_{2})$. So $\left| D_{n}^{*} \right| =\left| \left\langle f,g\right\rangle \right| =2n$. $D_{n}^{*}\subset \left\langle f,g\right\rangle$, so $D_{n}^{*}=\left\langle f,g\right\rangle$. Thus, $D_{n}^{*}\cong \left\langle f,g\right\rangle\cong D_{n}$.
    \end{enumerate}
\end{answer}