\section{Free modules and vector spaces}
\begin{ex}
    \begin{enumerate}[(a)]
        \item A set of vectors $\{x_{1},\cdots x_{n}\}$ in a vector space $V$ over a division ring $R$ is linearly dependent if and only if some $x_{k}$ is a linear combination of the preceding $x_{i}$.
        \item If $\{x_{1},x_{2},x_{3}\}$ is a linearly independent subset of $V$, then the set $\{x_{1}+x_{2},x_{2}+x_{3},x_{3}+x_{1}\}$ is linearly independent if and only if $\mathrm{Char}R\neq 2$.
    \end{enumerate}
\end{ex}

\begin{answer}
    \begin{enumerate}[(a)]
        \item If $x_{k}$ is a linear combination of preceding $x_{i}$, then $x_{k}=r_{1}x_{1}+\cdots+r_{k-1}x_{k-1}+r_{k+1}x_{k+1}+\cdots+ r_{n}x_{n}$ with some $r_{i}\neq 0$, so $r_{1}x_{1}+\cdots+r_{k-1}x_{k-1}-x_{k}+r_{k+1}x_{k+1}+\cdots +r_{n}x_{n}=0$. So $\{x_{1},\cdots,x_{n}\}$ is linearly dependent.
        
        Conversely, $r_{1}x_{1}+\cdots+r_{n}x_{n}=0$ for some non-trivial $r_{i}$. WLOG, assume $r_{1}\neq 0$, then $x_{1}=(-\frac{r_{2}}{r_{1}})x_{2}+(-\frac{r_{3}}{r_{1}})x_{3}+\cdots+(-\frac{r_{n}}{r_{1}})x_{n}$ is a linear combination.
        \item If $\mathrm{char} R\neq 2$, $r_{1}(x_{1}+x_{2})+r_{2}(x_{2}+x_{3})+r_{3}(x_{1}+x_{3})=(r_{1}+r_{3})x_{1}+(r_{1}+r_{2})x_{2}+(r_{2}+r_{3})x_{3}=0$. $\{x_{1},x_{2},x_{3}\}$ is linearly independent, so $r_{1}+r_{3}=r_{1}+r_{2}=r_{2}+r_{3}=0$, $2r_{1}+r_{2}+r_{3}=2r_{1}=0$, we have $r_{1}=0$. Similarly, $r_{2}=r_{3}=0$, $\{x_{1}+x_{2},x_{2}+x_{3},x_{1}+x_{3}\}$ is linearly independent.
        
        Conversely, suppose $\mathrm{char}R=2$, $\exists r_{1},r_{2},r_{3}\in R$, $2r_{1}=2r_{2}=2r_{3}=0$. We have $(r_{1}+r_{2}-r_{3})(x_{1}+x_{2})+(r_{1}+r_{3}-r_{2})(x_{1}+x_{3})+(r_{3}+r_{2}-r_{1})(x_{3}+x_{2})=2r_{1}x_{1}+2r_{2}x_{2}+2r_{3}x_{3}=0$, but $r_{1}+r_{2}-r_{3}$, $r_{1}+r_{3}-r_{2}$, $r_{3}+r_{2}-r_{1}$ is not necessary to be 0.
    \end{enumerate}
\end{answer}

$$ $$

\begin{ex}
    Let $R$ be any ring (possibly without identity) and $X$ a nonempty set. In this exercise an $R$-module $F$ is called a \textbf{free module on $X$} if $F$ is a free object on $X$ in the category of all left $R$-modules. Thus by Definition I.7.7m $F$ is the free module on $X$ if there is a function $\tau:X\to F$ such that for any left $R$-module $A$ and function $f:X\to A$ there is a unique $R$-module homomorphism $\bar{f}:F\to A$ with $\bar{f}\tau=f$.
    \begin{enumerate}[(a)]
        \item Let $\{X_{i}|i\in I\}$ be a collection of mutually disjoint sets and for each $i\in I$, suppose $F_{i}$ is a free module on $X_{i}$, with $\tau_{i}:X_{i}\to F_{i}$. Let $X=\bigcup\limits_{i\in I}X_{i}$ and $F=\sum\limits_{i\in I}F_{i}$, with $\phi_{i}:F_{i}\to F$ the canonical injection. Define $\tau:X\to F$ by $\tau(x)=\phi_{i}\tau_{i}(x)$ for $x\in X_{i}$. Prove that $F$ is a free module on $X$.
        \item Assume $R$ has an identity. Let the abelian group $\mathbf{Z}$ be given trivial $R$-module structure ($rm=0$ for all $r\in R, m\in\mathbf{Z}$), so that $R\oplus\mathbf{Z}$ is an $R$-module with $r(r',m)=(rr',0)$ for all $r,r'\in R$, $m\in \mathbf{Z}$. If $X$ is any one element set, $X=\{t\}$, let $\tau:X\to R\oplus\mathbf{Z}$ be given by $\tau(t)=(1_{R},1)$. Prove that $R\oplus\mathbf{Z}$ is a free module on $X$.
        \item If $R$ is an arbitrary ring and $X$ is any set, then there exists a free module on $X$.
    \end{enumerate}
\end{ex}

$$ $$

\begin{ex}
    Let $R$ be any ring (possibly without indentity) and $F$ a free $R$-module on the set $X$, with $\tau X:\to F$, as in \textbf{Exercise 4.2.2}. Show that $\tau(X)$ is a set of generators of the $R$-module $F$.
\end{ex}

$$ $$

\begin{ex}
    Let $R$ be a principal ideal domain, $A$ a unitary left $R$-module, and $p\in R$ a prime (= irreducible). Let $pA=\{pa|a\in A\}$ and $A[p]=\{a\in A|pa=0\}$.
    \begin{enumerate}[(a)]
        \item $R/(p)$ is a field.
        \item $pA$ and $A[p]$ are submodules of $A$.
        \item $A /pA$ is a vector space over $R /(p)$, with $(r+(p))(a+pA)=ra+pA$.
        \item $A[p]$ is a vector space over $R /(p)$, with $(r+(p))a=ra$.
    \end{enumerate}
\end{ex}

$$ $$

\begin{ex}
    Let $V$ be a vector space over a division ring $D$ and $S$ the set of all subspaces of $V$, partiallly ordered by set theoretic inclusion.
    \begin{enumerate}[(a)]
        \item $S$ is a complete lattice.
        \item $S$ is a complemented lattice; that is, for each $V_{1}\in S$ there exists $V_{2}\in S$ such that $V=V_{1}+V_{2}$ and $V_{1}\cap V_{2}=0$, so that $V=V_{1}\oplus V_{2}$.
        \item $S$ is modular lattice; that is, if $V_{1}, V_{2}, V_{3}\in S$ and $V_{3}\subset V_{1}$, then \[V_{1}\cap (V_{2}+V_{3})=(V_{1}\cap V_{2})+V_{3}\]
    \end{enumerate}
\end{ex}

$$ $$

\begin{ex}
    Let $\mathbf{R}$ and $\mathbf{C}$ be the fields of real and complex numbers respetively.
    \begin{enumerate}[(a)]
        \item $\mathrm{dim}_{\mathbf{R}}\mathbf{C}$ and $\mathrm{dim}_{\mathbf{R}}\mathbf{R}=1$.
        \item There is no field $K$ such that $\mathbf{R}\subset K\subset\mathbf{C}$.
    \end{enumerate}
\end{ex}

$$ $$

\begin{ex}
    If $G$ is a nontrivial group that is not cyclic of order 2, then $G$ has a nonidentity automorphism.
\end{ex}

$$ $$

\begin{ex}
    If $V$ is a finite dimensional vector space and $V^{m}$ is the vector space\[V\oplus V\oplus\cdots\oplus V \,(m \text{ summands})\] then for each $m\geq 1$, $V^{m}$ is finite dimensional and $\mathrm{dim} V^{m}=m(\mathrm{dim}V)$.
\end{ex}

$$ $$

\begin{ex}
    If $F_{1}$ and $F_{2}$ are free modules over a ring with the invariant dimension property, then $\mathrm{rank}(F_{1}\oplus F_{2})=\mathrm{rank}F_{1}+\mathrm{rank}F_{2}$.
\end{ex}

$$ $$

\begin{ex}
    Let $R$ be a ring with no zero divisors such that for all $r,s\in R$ there exists $a,b\in R$, not both zero, with $ar+bs=0$.
    \begin{enumerate}[(a)]
        \item If $R=K\oplus L$, then $K=0$ or $L=0$.
        \item If $R$ has an identity, then $R$ has the invariant dimension property.
    \end{enumerate}
\end{ex}

$$ $$

\begin{ex}
    Let $F$ be a free module of infinite rank $\alpha$ over a ring $R$ that has the invariant dimension property. For each cardinal $\beta$ such that $0\leq \beta\leq \alpha$, $F$ has infinitely many proper free submodules of rank $\beta$.
\end{ex}

$$ $$

\begin{ex}
    If $F$ is a free module ober a ring with indentity such that $F$ has a basis of finite cardinality $n\geq 1$ and another basis of cardinality $n+1$, then $F$ has a basis of cardinality $m$ for every $m\geq n$($m\in \mathbf{N}^{*}$).
\end{ex}

$$ $$

\begin{ex}
    Let $K$ be a ring with identity and $F$ a free $K$-module with an infinite demumerable basis $\{e_{1},e_{2},\dots\}$. Then $R=\mathrm{Hom}_{K}(F,F)$ is a ring by \textbf{Exercise 4.1.7}. If $n$ is any positive integer, then the free left $R$-module $R$ has a basis of $n$ elementsl that is,  as an $R$-module, $R\cong R\oplus\cdots\oplus R$ for any finite number of summands.
\end{ex}

$$ $$

\begin{ex}
    Let $f:V\to V'$ be a linear transformation of finite dimensional vector space $V$ and $V'$ such that $\mathrm{dim}V=\mathrm{dim}V'$. Then the following conditions are equivalent: (i) $f$ is an isomorphism; (ii) $f$ is an epimorphisml $f$ is a monomorphism.
\end{ex}

$$ $$

\begin{ex}
    Let $R$ be a ring with identity. Show that $R$ is not a free module on any set in the category of all $R$-modules.
\end{ex}