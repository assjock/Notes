\section{Ideals}
\begin{ex}
    The set of all nilpotent elements in a commutative ring forms an ideal.
\end{ex}

\begin{answer}
    Assume the set is $I$, then $\forall a,b\in I$, $a^{m}=b^{n}=0$, $(a+b)^{m+n}=0$ and $(ab)^{mn}=0$ so $a+b\in I$, $ab\in I$. $I$ is a subring. $\forall x\in R$, $(xa)^{m}=x^{m}a^{m}=0$, $(ax)^{m}=a^{m}x^{m}=0$, so $xa\in I$ and $ax\in I$, $I$ is an ideal.
\end{answer}

$$ $$

\begin{ex}
    Let $I$ be an ideal in a commutative ring $R$ and let $\mathrm{Rad} I=\{r\in R| r^{n}\in I \text{ for some } n\}$. Show that $\mathrm{Rad}I$ is an ideal.
\end{ex}

\begin{answer}
    $\mathrm{Rad} I$ is a ring since $R$ is a commutative ring. For $r\in \mathrm{Rad} I$ and $\forall x\in R$, $(xr)^{n}=x^{n}r^{n}\in I$ so $xr\in \mathrm{Rad} I$, $(rx)^{n}=r^{n}x^{n}\in I$ so $rx\in \mathrm{Rad} I$. Thus $\mathrm{Rad} I$ is an ideal.
\end{answer}

$$ $$

\begin{ex}
    If $R$ is a ring and $a\in R$, then $J=\{r\in R|ra=0\}$ is a left ideal and $K=\{r\in R|ar=0\}$ is a right ideal in $R$.
\end{ex}

\begin{answer}
    $J$ is a subring of $R$. For $r\in J$ and $\forall x\in R$, $(xr)a=x(ra)=0$ so $xr\in J$, $J$ is a left ideal. Similarly, $I$ is a right ideal.
\end{answer}

$$ $$

\begin{ex}
    If $I$ is a left ideal of $R$, then $A(I)=\{r\in R|rx=0 \text{ for every }x\in I\}$ is an ideal in $R$.
\end{ex}

\begin{answer}
    For any $a,b\in A(I)$, we have $ab\in A(I)$ and $a+b\in A(I)$. For $r\in A(I)$ and $\forall x\in R$, $(xr)x'=x(rx')=0$ for every $x'\in I$, so $xr\in A(I)$. $(rx)x'=r(xx')$, $x x'\in I$ so $rx x'=0$, $rx\in A(I)$. Thus $A(I)$ is an ideal of $R$.
\end{answer}

$$ $$

\begin{ex}
    If $I$ is an ideal in a ring $R$, let $\left[R:I\right]=\{r\in R|xr\in I \text{ for every }x\in R\}$. Prove that $\left[ R:I\right]$ is an ideal of $R$ which contains $I$.
\end{ex}

\begin{answer}
    $I$ is a subring of $R$ so $\left[R:I\right]$ is also a subring of $R$. For $r\in\left[R:I\right]$ and $x, x'\in R$, $x'xr=(x'x)r\in I$ so $xr\in \left[R:I\right]$, $x'rx=(x'r)x\in I$ so $rx\in \left[R:I\right]$. $\left[R:I\right]$ is an ideal of $R$. Since $\forall r\in I$, $xr\in I$ and $rx\in I$, $I\subset \left[R:I\right]$.
\end{answer}

$$ $$

\begin{ex}
    \begin{enumerate}[(a)]
        \item The center of the ring $S$ of all $2\times 2$ matrices over a field $F$ consists of all matrices of the form $\begin{pmatrix}
            a&0\\0&a
        \end{pmatrix}$.
        \item Then center of $S$ is not an ideal in $S$.
        \item What is the center of the ring of all $n\times n$ matrices over a division ring?
    \end{enumerate}
\end{ex}

\begin{answer}
    \begin{enumerate}[(a)]
        \item $\forall x\in M_{F}(2,2)$, $x=\begin{pmatrix}
            x_{1}&x_{2}\\x_{3}&x_{4}
        \end{pmatrix}$\[x\begin{pmatrix}
            a&0\\0&a
        \end{pmatrix}=\begin{pmatrix}
            a&0\\0&a
        \end{pmatrix}x=\begin{pmatrix}
            ax_{1}&ax_{2}\\ax_{3}&ax_{4}
        \end{pmatrix}\] so $\begin{pmatrix}
            a&0\\0&a
        \end{pmatrix}\in C(M_{F}(2,2))$.

        $\forall \begin{pmatrix}
            a_{1}&a_{2}\\a_{3}&a_{4}
        \end{pmatrix}\in C(M_{F}(2,2))$, take $\begin{pmatrix}
            0&1_{F}\\0&0
        \end{pmatrix}\in M_{F}(2,2)$\[\begin{pmatrix}
            a_{1}&a_{2}\\a_{3}&a_{4}
        \end{pmatrix}\begin{pmatrix}
            1_{F}&0\\0&0
        \end{pmatrix}=\begin{pmatrix}
            a_{1}&0\\a_{3}&0
        \end{pmatrix}\]\[\begin{pmatrix}
            1_{F}&0\\0&0
        \end{pmatrix}\begin{pmatrix}
            a_{1}&a_{2}\\a_{3}&a_{4}
        \end{pmatrix}=\begin{pmatrix}
            a_{1}&a_{2}\\0&0
        \end{pmatrix}\] so $a_{2}=a_{3}=0$.
        \[\begin{pmatrix}
            a_{1}&a_{2}\\a_{3}&a_{4}
        \end{pmatrix}\begin{pmatrix}
            0&1_{F}\\0&0
        \end{pmatrix}=\begin{pmatrix}
            0&a_{1}\\0&a_{3}
        \end{pmatrix}\]\[\begin{pmatrix}
            0&1_{F}\\0&0
        \end{pmatrix}\begin{pmatrix}
            a_{1}&a_{2}\\a_{3}&a_{4}
        \end{pmatrix}\begin{pmatrix}
            a_{3}&a_{4}\\0&0
        \end{pmatrix}\] so $a_{1}=a_{4}$. All the elements of $C(M_{F}(2,2))$ has the form $\begin{pmatrix}
            a&0\\0&a
        \end{pmatrix}$.
        \item For $c\in C(S)$. If $S$ is not commutative, $\forall x,x'\in R$, we need $xc\in C(S)\Rightarrow x'xc=xcx'=xx'c$, however, this may not always true.
        \item By multiplying $\begin{pmatrix}
            1_{F}& &\\ &\ddots & \\ & &0
        \end{pmatrix}$, $\begin{pmatrix}
            0& & &\\ &1_{F}& &\\ & &\ddots&\\ & & &0
        \end{pmatrix}$, $\dots$, $\begin{pmatrix}
            0& &\\ &\ddots & \\ & &1_{F}
        \end{pmatrix}$, we can have $C(M_{F}(2,2))$ consist of all the elements in the form of $a\begin{pmatrix}
            1_{F}& & &\\ &1_{F}& &\\ & &\ddots&\\ & & &1_{F}
        \end{pmatrix}$.
    \end{enumerate}
\end{answer}
$$ $$

\begin{ex}
    \begin{enumerate}[(a)]
        \item A ring $R$ with identity is a division ring if and only if $R$ has no proper left ideals.
        \item If $S$ is a ring (possibly without identity) with no proper left ideals, then either $S^{2}=0$ or $S$ is a division ring.
    \end{enumerate}
\end{ex}

\begin{answer}
    \begin{enumerate}[(a)]
        \item Suppose not. $I$ is an ideal in $R$. $\forall r\in I$, take $r^{-1}\in R$, then $1_{R}\in I$ so $I=R$ is not a  proper ideal.
        \item $I=\{a\in S|Sa=0\}$ is a left ideal since $\forall x,x'\in S$, $x'(xs)=(x'x)s=0$, $xs\in I$. Thus $I=0$ or $I=S$. If $I=S$, then $S^{2}=0$. If $I=0$, we prove $S$ has no zero divisor.
        
        For the set $I'=\{r\in S|rb=0\}$, $I'\subset I$. $I'$ is a subring of $S$, and $I'$ is also a left ideal of $S$. So $I'=0$, $b$ has no left zero divisors. $\forall a\in S$, $Sa$ is a left ideal of $S$. $Sa\neq 0$ so $Sa=S$. Thus, $\exists 1_{S}\in S$, such that $1_{S}a=a$. Since $s_{1}-s_{2}$ has no left zero divisor, $as_{1}=as_{2}\Rightarrow s_{1}=s_{2}$. So $aS=S$. For all $s\in S$, $\exists s'$ s.t. $s=as'$ so $\forall s\in S$, $1_{S}\cdot s=1_{S}as'=as'=s$. $aS=S$ so $\exists 1_{S}'\in S$, $a1_{S}'=a$. Similarly, $\forall s\in S$, $s1_{S}=s$. Then $1_{S}1_{S}'=1_{S}=1_{S}'$ so $S$ has identity. Since $Sa=aS=S$, we can have $S$ is a division ring.
    \end{enumerate}
\end{answer}

$$ $$

\begin{ex}
    Let $R$ be a ring with identity and $S$ the ring of all $n\times n$ matrices over $R$. $J$ is an ideals of $S$ if and only if $J$ is the ring of all $n\times n$ matrices over $I$ for some ideal $I$ in $R$.
\end{ex}

\begin{answer}
    If $J$ is an ideal. Denote $E_{r,s}$ as the matrix which has $1_{R}$ as the $r$ column and $s$ row. Then $\forall A=(a_{ij})$, $E_{p,r}AE_{s,q}$ is a matrix with $a_{rs}$ in the $p$ column and  $q$ row. So for $A\in J$ $(aE_{p,r})A(bE_{s,q})$ is the matrix with $aa_{rs}b$ in the $p$ column and  $q$ row. $aa_{rs}b\in I$. Then because of closure we know $J$ contains all $n\times n$ matrices over $I$. 

    If $J$ consists of all $n\times n$ matrices over $I$, the proof is trivial.
\end{answer}

$$ $$

\begin{ex}
    Let $S$ be the ring of all $n\times n$ matrices over a division ring $D$.
    \begin{enumerate}[(a)]
        \item $S$ has no proper ideals (that is, 0 is the maximal ideal).
        \item $S$ has zero divisors. Consequently, (i) $S\cong S /0$ is not a division ring and (ii) 0 is a prime ideal which does not satisfy condition (1) of Theorem 2.15.
    \end{enumerate}
\end{ex}

\begin{answer}
    \begin{enumerate}[(a)]
        \item $J$ is an ideal of $S$ so $J$ consists of all $n\times n$ matrices over $I$ where $I$ is an ideal of $D$. From \textbf{Exercise 3.2.7}, $D$ has no proper ideal so $I=0\Rightarrow J=0$.
        \item For $A=(a_{ij})$ with $a_{ri}=0$ for $i=1,2\cdots$ and other entries doesn't equals to zero, we have $E_{1r}A=0$. $S$ has no zero divisors.
    \end{enumerate}
\end{answer}

$$ $$

\begin{ex}
    \begin{enumerate}[(a)]
        \item Show that $\mathbf{Z}$ is a principal ideal ring.
        \item Every homomorphic image of a principal ideal ring is also a principal ideal ring.
        \item $Z_{m}$ is a principal ideal ring for every $m>0$.
    \end{enumerate}
\end{ex}

\begin{answer}
    \begin{enumerate}[(a)]
        \item For any ideal $I$ in $\mathbf{Z}$, $I$ is a subring so $I=m\mathbf{Z}$ where $m\in \mathbf{Z}$. $m\mathbf{Z}=(m)$ is a principal ideal so $\mathbf{Z}$ is a PID.
        \item For $f:R\to S$ with $f(r)=s$ and $R$ is a principal ideal ring. Consider $f:R\to \mathrm{Im}f\subset S$. For any ideal $J\subset \mathrm{Im}f$, $f^{-1}(J)$ is an ideal since $\forall a\in f^{-1}(J)$ and $r\in R$, $f(ar)=f(a)f(r)\in J\Rightarrow ar\in f^{-1}(J)$. $f^{-1}(J)$ is a principal ideal, assume $f^{-1}(J)=(a)$. Then $\forall r\in R$, $ar\in (a)$, $ra\in (a)$. $f(ar)=f(a)f(r)\in J$ and $f(ra)=f(r)f(a)\in J$ since $f(a)\in J$ and $f(r)\in S$. So $(f(a))\subset J$. $J=f((a))=\{f(ra+as+na+\sum\limits_{i=1}^{m}r_{i}as_{i})| r, s, r_{i}, s_{i}\in R, n\in \mathbf{Z}\}=\{f(r)f(a)+f(a)f(s)+nf(a)+\sum\limits_{i=1}^{m}f(r_{i})f(a_{i})f(s_{i})| r, s, r_{i}, s_{i}\in R, n\in \mathbf{Z}\}\subset (f(a))$. So $J=(f(a))$ is a principal ideal. The image of a principal ideal ring is also a principal ideal ring.
    \end{enumerate}
\end{answer}

$$ $$

\begin{ex}
    If $N$ is the ideal of all nilpotent elements in a commutative ring $R$, then $R /N$ is a ring with no nonzero nilpotent elements.
\end{ex}

\begin{answer}
    Suppose not. $\exists r\in R$, $r\notin N$, $(r+N)^{n}=0$ for some $n\in\mathbf{N}$. \[(r+N)^{n}=r^{n}+N=N\Rightarrow r^{n}\in N\]so for some $m\in \mathbf{N}$, $r^{nm}=0\Rightarrow r\in N$. That's contradictory!
\end{answer}

$$ $$

\begin{ex}
    Let $R$ be a ring without identity and with no zero divisors. Let $S$ be the ring whose additive group is $R\times \mathbf{Z}$ as in the proof of Theorem 1.10. Let $A=\{(r,n)\in S|rx+nx=0 \text{ for every }x\in R\}$.
    \begin{enumerate}[(a)]
        \item $A$ is an ideal in $S$.
        \item $S /A$ has an identity and contains a subring isomorphic to $R$.
        \item $S /A$ has no zero divisors.
    \end{enumerate}
\end{ex}

\begin{answer}
    \begin{enumerate}[(a)]
        \item For $(r,n), (r',n')\in S$, $(r'+r)x+(n'+j)x=rx+nx+r'x+n'x=0$, so $(r+r',n+n')\in A$. $(r,n)(r'n')=(rr'+nr'+n'r,nn')$, $rr'x+n'rx+nr'x+nn'x=r(r'x+n'x)+n(r'x+n'x)=0$, so $(r,n)(r',n')\in A$. $A$ is a subring of $R\times \mathbf{Z}$. $\forall (r_{1},n_{1})\in R\times \mathbf{Z}$, $(r_{1},n_{1})(r,n)=(r_{1}r+nr_{1}+n_{1}r,nn_{1})\Rightarrow r_{1}rx+nr_{1}x+n_{1}rx+nn_{1}x=r_{1}(rx+nx)+n_{1}(rx+nx)=0\Rightarrow (r_{1},n_{1})(r,n)\in A$. $A$ is an ideal of $R\times \mathbf{Z}$.
        \item Take $0_{R}\in R$ and $(0_{R},1)\in S$. Then $(0_{R},1)+A$ is an identity of $S /A$. \[\forall (r,n)\in S, \quad (r,n)(0_{R},1)=(0_{R},1)(r,n)=(r,n)\]
        \item For any $(r,n), (s,m)$ satisfy that $(r,n)(s,m)\in A$, we prove that $(r,n)\in A$ or $(s,m)\in A$. Suppose $sx+mx\neq 0$, $r(sx+mx)+n(sx+mx)=0\Rightarrow (sx+mx)r(sx+mx)+n(sx+mx)^{2}=0\Rightarrow ((sx+mx)r+n(sx+mx))(sx+mx)=0\Rightarrow (sx+mx)r+n(sx+mx)=0$. For any $x\in R$, $(sx+mx)rx+n(sx+mx)x=0\Rightarrow(sx+mx)(rx+nx)=0\Rightarrow rx+nx=0$, so $(r,n)\in A$. $S /A$ has no divisor.
    \end{enumerate}
\end{answer}

$$ $$

\begin{ex}
    Let $f:R\to S$ be a homomorphism of rings, $I$ and ideal in $R$, and $J$ an ideal in $S$.
    \begin{enumerate}[(a)]
        \item $f^{-1}(J)$ is and ideal in $R$ that contains $\mathrm{Ker}f$.
        \item If $f$ is an epimorphism, then $f(I)$ is an ideal in $S$. If $f$ is not surjective, $f(I)$ need not be an ideal.
    \end{enumerate}
\end{ex}

\begin{answer}
    \begin{enumerate}[(a)]
        \item $\forall a\in f^{-1}(J)$ and $r\in R$, $f(ar)=f(a)f(r)\in J\Rightarrow ar\in J$. Similarly, $ra\in J$, $f^{-1}(J)$ is an ideal. $\mathrm{Ker}f\subset f^{-1}(J)$ since $0_{S}\in J$.
        \item $\forall b\in f(I)$ and $s\in S$, $f$ is a epimorphism so $s=f(r)$, $b=f(a)$ for some $r, a\in R$. $sb=f(r)f(a)=f(ar)$, $ar\in I\Rightarrow sb\in f(I)$, similarly $bs\in f(I)$. $f(I)$ is an ideal.
        
        If $f$ is not surjective. Take $Z\left[x\right]$ and $\mathbf{Z}$ which is a subring but not an ideal in $Z\left[x\right]$. $\mathbf{Z}$ is an ideal of itself, $f=1_{\mathbf{Z}}$ satisfies the condition.
    \end{enumerate}
\end{answer}

$$ $$

\begin{ex}
    If $P$ is an ideal in a not necessarily commutative ring $R$, then the following conditions are equivalent.
    \begin{enumerate}[(a)]
        \item $P$ is a prime ideal.
        \item If $r,s\in R$ are such that $rRs\subset R$, then $r\in P$ or $s\in P$.
        \item If $(r)$ and $(s)$ are principal ideals of $R$ such that $(r)(s)\subset P$, then $r\in P$ or $s\in P$.
        \item If $U$ and $V$ are right ideals in $R$ such that $UV\subset R$, then $U\subset R$ or $V\subset R$.
        \item If $U$ and $V$ are left ideals in $R$ such that $UV\subset R$, then $U\subset R$ or $V\subset R$.
    \end{enumerate}
\end{ex}

$$ $$

\begin{ex}
    The set consisting of zero and all zero divisors in a commutative ring with identity contains at least one prime ideal.
\end{ex}

\begin{answer}
    Denote $S=R-Z$. $\forall a,b\in S$, we prove that $ab\in S$. Suppose $\exists (ab)c=0$ for some $c\in R$, $a$, $b$ are not zero divisors so $abc=b(ac)=a(bc)=0$, so $ac=0$, $bc=0\Rightarrow c=0$, so $ab$ is not a zero divisor. Thus $Z=R-S$ contains an prime ideal.
\end{answer}

$$ $$

\begin{ex}
    Let $R$ be a commutative ring with identity and suppose that the ideal $A$ of $R$ is contained in a finite union of prime ideals $P_{1}\cup\cdots\cup P_{n}$. Show that $A\subset P_{i}$ for some $i$.
\end{ex}

\begin{answer}
    Suppose not. We choose the smallest $I$ such that for all $i\in I$, $P_{i}\cap A\neq \varnothing$ and $A\cap P_{i}\not\subset\bigcup\limits_{j\neq i}P_{j}$ for any $i\in I$. So $\exists a_{i}\in (A\cap P_{i})-(\bigcup\limits_{j\neq i}P_{j})$, $\forall i\in I$. Take $x=a_{1}+a_{2}a_{3}\cdots a_{n}$, $x\in A$ since $a_{i}\in A$ for all $i\in I$. And $x\notin P_{i}$ for $i=2,3,\dots, n$ since $a_{1}\notin P_{i}$, $i=2,3,\dots, n$. $x\notin P_{1}$ since $P_{1}$ is prime and $a_{2}, \dots, a_{n}\notin P_{1}$. So $x\notin \bigcup\limits_{j\neq i}P_{j}$, which is contradictory!
\end{answer}

$$ $$

\begin{ex}
    Let $f:R\to S$ be an epimorphism of rings with kernel $K$.
    \begin{enumerate}[(a)]
        \item If $P$ is a prime ideal in $R$ that contains $K$, then $f(P)$ is a prime ideal in $S$.
        \item If $Q$ is a prime ideal in $S$, then $f^{-1}(Q)$ is a prime ideal in $R$ that contains $K$.
        \item There is a one-to-one correspondence between the set of all prime ideals in $R$ that contain $K$ and the set of all prime ideals in $S$, given by $P\mapsto f(P)$.
        \item If $I$ is an ideal in a ring $R$, then every prime ideal in $R /I$ is of the form $P /I$, where $P$ is a prime ideal in $R$ that contains $I$.
    \end{enumerate}
\end{ex}

\begin{answer}
    \begin{enumerate}[(a)]
        \item From \textbf{Exercise 3.2.13} we know $f(P)$ is an ideal. $\forall x,y\in f(P)$, $\exists a.b\in R$, $x=f(a)$, $y=f(b)$ and $a,b\notin P$. Assume $\exists p\in P$ such that $f(ab)=f(p)$, then $f(ab-p)=0$, $ab-p\in\mathrm{Ker}f\subset P\Rightarrow ab\in P$. That's contradictory to $a,b\notin P$ so $xy\notin f(P)$. $f(P)$ is prime.
        \item From \textbf{Exercise 3.2.13}, $f^{-1}(Q)$ is an ideal. Take $g:S\to S /Q$ and $gf:R\to S /Q$. By the Theorem of homomorphism, $R /f^{-1}(Q) \cong S /Q$ is a ring without divisor, so $f^{-1}(Q)$ is prime.
        \item From (a), (b), $f$ is a one-to-one map between prime ideals given by $P\mapsto f(P)$.
        \item Consider the homomorphism $f:R\to R /I$. For any prime ideal $P\subset R$ and $f(P)$ is an prime ideal in $R$, $\mathrm{Ker}f=I$ so for prime ideals $I\subset P\subset R$. $P$ can have one to one correspondence with $f(P)=P /I\subset R/I$. So all the prime ideals has the form $P /I$.
    \end{enumerate}
\end{answer}

$$ $$

\begin{ex}
    An ideal $M\neq R$ in a commutative ring $R$ with identity is maximal if and only if for every $r\in R-M$, there exists $x\in R$ such that $1_{R}-rx\in M$.
\end{ex}

\begin{answer}
    If $M$ is maximal, then $M$ is prime. So $rR+M=R$, $r(R-M)+M=R$ and $r(R-M)\cap M=\varnothing$. Take $1_{R}\in R$ we have $x\in R-M$, $1_{R}-xr\in M$. If $\forall r\in R-M$, $\exists x\in R$ such that $1_{R}-rx\in M$. Suppose $M\subset I\subset R$ where $I$ is an ideal, $I\neq R$ so $1_{R}\notin R$. Take $r\in I-M\subset R-M$, then $\forall x\in R$, $rx\in I$, so $1_{R}-rx\notin I$ thus $1_{R}-rx\notin M$. That's contradictory!
\end{answer}

$$ $$

\begin{ex}
    The ring $E$ of even integers contains a maximal ideal $M$ such that $E /M$ is not a field.
\end{ex}

\begin{answer}
    $E=2\mathbf{Z}$ and $M$ is a maximal ideal in $E$ and for any subring of $E$ has the form $wn\mathbf{Z}$ where $n\in \mathbf{Z}$. $2n\mathbf{Z}$ is an ideal in $2\mathbf{Z}$. Take $n=15$, $(2,15)=1$ so $2\mathbf{Z} /30\mathbf{Z}\cong \mathbf{Z} /15\mathbf{Z}$ which is not a field since $3\cdot 5=0$ is a zero divisor.
\end{answer}

$$ $$

\begin{ex}
    In the ring $\mathbf{Z}$ the following conditions on a nonzero ideal $I$ are equivalent: (i) $I$ is prime; (ii) $I$ is maximal; (iii) $I=(p)$ with $p$ prime.    
\end{ex}

\begin{answer}
    $\mathbf{Z}$ is an integer domain so (ii)$\Rightarrow$(i).

    (i)$\Rightarrow$(iii): Trivial.

    (iii)$\Rightarrow$(ii): For any $n\notin (p)$, we have $p\nmid n$ thus $\exists x,y\in \mathbf{Z}$ such that $px+my=1$. Consider an ideal $I$ and $(p)\subset I$, $n\in I$, then $1\in I$ so $I=\mathbf{Z}$ which means $(p)$ is maximal.
\end{answer}

$$ $$

\begin{ex}
    Determine all prime and maximal ideals in the ring $Z_{m}$.
\end{ex}

\begin{answer}
    $Z_{m}^{2}=Z_{m}$ so every maximal ideal is prime in $Z_{m}$. $Z_{m}\cong \mathbf{Z} /m\mathbf{Z}$ via $\varphi:\bar{x}\mapsto mz+x$. From \textbf{Exercise 3.2.17}, all the prime ideals in $\mathbf{Z}/m\mathbf{Z}$ are $P /m\mathbf{Z}$, where $P$ is a prime ideal contains $m\mathbf{Z}=(m)$.

    If $m$ is prime, $(m)$ is prime, too. So no such ideal exist.

    If $m=p_{1}^{s_{1}}p_{2}^{s_{2}}\cdots p_{n}^{s_{n}}$ where $p_{i}$ are primes, then $(p_{1}), (p_{2}),\dots, (p_{n})$ are prime ideals and $f((\bar{p_{i}}))=(p_{i})/m\mathbf{Z}$ are prime ideals. So all the prime ideals in $Z_{m}$ are $(\bar{p_{i}})$, $i,1,2,\dots,n$.
\end{answer}

$$ $$

\begin{ex}
    \begin{enumerate}[(a)]
        \item If $R_{1},\dots, R_{n}$ are rings with identity and $I$ is an ideal in $R_{1}\times \cdots\times R_{n}$, then $I=A_{1}\times \cdots\times A_{m}$, where each $A_{i}$ is an ideal in $R_{i}$.
        \item Show that the conclusion of (a) need not hold if the rings $R_{i}$ do not have identities.
    \end{enumerate}
\end{ex}

$$ $$

\begin{ex}
    An element $e$ in a ring $R$ is said to be \textbf{idempotent} if $e^{2}=e$. An element  of the center of the ring $R$ is said to be central. If $e$ is a central idempotent in a ring $R$ with identity, then
    \begin{enumerate}[(a)]
        \item $1_{R}-e$ is a central idempotent;
        \item $eR$ and $(1_{R}-e)R$ are ideals in $R$ such that $R=eR\times (1_{R}-e)R$.
    \end{enumerate}
\end{ex}

\begin{answer}
    \begin{enumerate}[(a)]
        \item $(1_{R}-e)^{2}=1_{R}-2e+e^{2}=1_{R}-2e+e=1_{R}-e$. $\forall x\in R$, $ex=xe$ so $(1_{R}-e)x=x-ex=x-xe=x(1_{R}-e)$. $1_{R}-e$ is a central idempotent.
        \item $eR\cup (1_{R}-e)R\subset R$ so $\left\langle eR\cap(1_{R}-e)R\right\rangle\subset R$. $R=eR+(1_{R}-e)R$ so $R\subset\left\langle eR\cap(1_{R}-e)R\right\rangle$. So $R=\left\langle eR\cap(1_{R}-e)R\right\rangle$. $\left\langle eR\right\rangle=eR$ and $\left\langle (1_{R}-e)R\right\rangle=(1_{R}-e)R$ so $\left\langle eR\right\rangle\cap \left\langle (1_{R}-e)R\right\rangle=0$. Thus $R=eR\times (1_{R}-e)R$.
    \end{enumerate}
\end{answer}

$$ $$

\begin{ex}
    Idempotent elements $e_{1},\dots,e_{n}$ in a ring $R$ are said to be \textbf{orthogonal} if $e_{i}e_{j}=0$ for $i\neq j$. If $R,R_{1},\dots, R_{n}$ are rings with identity, then the following conditions are equivalent:
    \begin{enumerate}[(a)]
        \item $R\cong R_{1}\times \cdots\times R_{n}$.
        \item $R$ contains a set of orthogonal central idempotents $\{e_{1}, \dots, e_{n}\}$ such that $e_{1}+e_{2}+\cdots+e_{n}=1_{R}$ and $e_{i}R\cong R$ for each $i$.
        \item $R$ is the internal direct product $R=A_{1}\times \cdots\times A_{n}$ where each $A_{i}$ is an ideal of $R$ such that $A_{i}\cong R_{i}$.
    \end{enumerate}
\end{ex}

\begin{answer}
    Assume $f:R_{1}\times \cdots\times R_{n}\to R$ is an isomorphism.

    (a)$\Rightarrow$(b): Denote $\bar{e_{1}}=(1_{R},0,\dots,0)$, $\bar{e_{2}}=(0,1_{R},\dots,0)$, $\dots$, $\bar{e_{n}}=(0, 0, \dots,\\1_{R})$. They are orthogonal central idempotent in $S=R_{1}\times \cdots\times R_{n}$ and $f(\bar{e_{n}})=e_{n}$, $e_{1}+e_{2}+\cdots+e_{n}=1_{S}$, $\sum\limits_{i=1}^{n}e_{i}S=S$.
    
    Take  $\varphi_{i}:(r_{1}, r_{2},\dots, r_{i}, \dots, r_{n})\mapsto r_{i}$. Then $\varphi_{i}$ is a well defined isomorphism between $e_{i}S$ and $R_{i}$. $e_{i}R\cong \bar{e_{i}}S\cong R_{i}$.

    (b)$\Rightarrow$(c): Take $A_{i}=e_{i}R$, then $A_{i}\cong R_{i}$. We need to prove $R=e_{i}R\times e_{2}R\times \cdots\times e_{n}R$. $e_{i}R\cap (e_{1}R+e_{2}R+\cdots+e_{i-1}R+e_{i+1}R+\cdots+e_{n}R)=0$ since $e_{i}x_{i}=e_{1}x_{1}+e_{2}x_{2}+\cdots+e_{i-1}x_{i-1}+e_{i+1}x_{i+1}+\cdots+e_{n}x_{n}\Rightarrow e_{i}^{2}x_{i}=0$. $R=1_{R}R=\sum\limits_{i=1}^{n}e_{i}R$ so $R=e_{i}R\times e_{2}R\times \cdots\times e_{n}R$.

    (c)$\Rightarrow$(a): Trivial.
\end{answer}

$$ $$

\begin{ex}
    If $m\in \mathbf{Z}$ has a prime decomposition $m=p_{1}^{k_{1}}\cdots p_{t}^{k_{t}}$($k_{i}>0$; $p_{i}$ distinct primes), then there is an isomorphism of rings $Z_{m}\cong Z_{p_{1}^{k_{1}}}\times \cdots\times Z_{p_{t}^{k_{t}}}$.
\end{ex}

\begin{answer}
    For any $m\in\mathbf{Z}$, $\mathbf{Z} /m\mathbf{Z}\cong Z_{m}$. $p_{1}^{k_{1}}\mathbf{Z}\cap\cdots\cap p_{t}^{k_{t}}\mathbf{Z}=m\mathbf{Z}$. So $\exists \varphi:Z_{m}\mapsto Z_{p_{i}^{k_{1}}}\times \cdots\times Z_{p_{t}^{k_{t}}}$. $\forall i,j\in I$, $p_{i}^{k_{i}}\in p_{i}^{k_{i}}\mathbf{Z}$ and $p_{j}^{k_{j}}\in p_{j}^{k_{j}}\mathbf{Z}$, $\exists x,y\in \mathbf{Z}$ such that $xp_{i}^{k_{i}}+yp_{j}^{k_{j}}=1\in\mathbf{Z}$. So $p_{i}^{k_{i}}\mathbf{Z}+p_{j}^{k_{j}}\mathbf{Z}=\mathbf{Z}$, $\varphi$ is an isomorphism so $Z_{m}\cong Z_{p_{1}^{k_{1}}}\times \cdots\times Z_{p_{t}^{k_{t}}}$.
\end{answer}

$$ $$

\begin{ex}
    If $R=\mathbf{Z}$, $A_{1}=(6)$ and $A_{2}=(4)$, then the map $\theta :R /A_{1}\cap A_{2}\to R_{1} /A_{1}\times R_{2} /A_{2}$ of Corollary 2.27 is not surjective.
\end{ex}

\begin{answer}
    $R /(A_{1}\cap A_{2})=Z_{12}$, $R /A_{1}=Z_{6}$ and $R /A_{2}=Z_{4}$. $\left| Z_{6}\times Z_{4} \right| =\left| Z_{6} \right| \times \left| Z_{4} \right| =24$ but $\left| Z_{12} \right| =12$, so $\theta$ is surjective.
\end{answer}