\section{Cosets and counting}
\begin{ex}
    Let $G$ be a group and $\{H_{i}|i\in I\}$ a family of subgroups. Then for any $a\in G$, $(\bigcap\limits_{i}H_{i})a=\bigcap\limits_{i}H_{i}a$.
\end{ex}

\begin{answer}
    $\bigcap\limits_{i}H_{i}$ is a subgroup of $G$. Take $x\in \bigcap\limits_{i}H_{i}$, $x\in H_{i}$, $\forall i\in I$. Then $xa\in H_{i}a$, $\forall i\in I$, so $xa\in \bigcap\limits_{i}(H_{i}a)$. Thus, $(\bigcap\limits_{i}H_{i})a=\bigcap\limits_{i}(H_{i}a)$.
\end{answer}

$$ $$

\begin{ex}
    \begin{enumerate}[(a)]
        \item Let $H$ be the cyclic subgroup (of order 2) of $S_{3}$ generated by $\begin{pmatrix}
            1 & 2 &3\\2& 1&3
        \end{pmatrix}$. Then no left cosets of $H$ (except $H$ itself) is also a right coset. There exists $a\in S_{3}$ such that $aH\cap Ha=\{a\}$.
        \item If $K$ is the cyclic subgroup (of order 3) of $S_{3}$ generated by $\begin{pmatrix}
            1 & 2&3\\2& 3 &1
        \end{pmatrix}$, then every left coset of $K$ is also a right coset of $K$.
    \end{enumerate}
\end{ex}

\begin{answer}
    \begin{enumerate}[(a)]
        \item $H=\{(12),(1)\}$. $S_{3}=\{(12),(13),(23),(1),(123),(132)\}$. For $a\in H$, $aH=Ha=H$.
        
        $a=(13)$, $aH=\{(13),(123)\}$, $Ha=\{(13),(132)\}$.

        $a=(23)$, $aH=\{(23),(132)\}$, $Ha=\{(23),(123)\}$.

        $a=(123)$, $aH=\{(123),(23)\}$, $Ha=\{(132),(13)\}$.

        $a=(132)$, $aH=\{(132),(13)\}$, $Ha=\{(123),(23)\}$.
        
        \item $K=\{(123),(132),(1)\}$. For $a\in K$, $aK=Ka=K$.
        
        $a=(12)$, $aK=Ka=\{(12),(23),(13)\}$.
        
        $a=(13)$, $aK=Ka=\{(12),(23),(13)\}$.

        $a=(23)$, $aK=Ka=\{(12),(23),(13)\}$.
    \end{enumerate}
\end{answer}

$$ $$

\begin{ex}
    The following conditions on a finite group $G$ are equivalent.
    \begin{enumerate}[(i)]
        \item $\left| G \right| $ is prime.
        \item $G\neq \left\langle e\right\rangle$ and $G$ has no proper subgroups.
        \item $G\cong Z_{p}$ for some prime $p$.
    \end{enumerate}
\end{ex}

\begin{answer}
    (i)$\Rightarrow$(ii): If there exists $S<G$, $S\neq G$, then $\left| S \right| | \left| G \right| =p$. That's contradictory!

    (ii)$\Rightarrow$(iii): $\forall a\in G$, take $S=\{na|n=1,2,\dots,p\}$. If there exists $ma=na,(1\leq m<n\leq p)$, $(n-m)a=0$. So there exists subgroup $S$, and $\left| S \right|=n-m <p$. That's contradictory! So $S<G$, $\left| S \right| =\left| G \right| \Rightarrow S=G\cong Z_{p}$.

    (iii)$\Rightarrow$(i): Trivial.
\end{answer}

$$ $$

\begin{ex}
    Let $a$ be an integer and $p$ be a prime such that $p\nmid a$. Then $a^{p-1}\equiv 1\mod p$.
\end{ex}

\begin{answer}
    $(Z_{p}\backslash\{\bar{0}\},\times)$ is a group of order $p-1$. From \textbf{Exercise 1.1.7}, we know that $\forall \bar{a}\in Z_{p}\backslash\{\bar{0}\}$ and $b\in Z_{p}\backslash\{\bar{0}\}$, taking different $\bar{b}$ we will have different $\bar{ab}\in Z_{p}\backslash\{\bar{0}\}$. $\bar{ab}$ travels through all the elements in $Z_{p}\backslash\{\bar{0}\}$. So \[\prod_{i=1}^{p-1}(\bar{i}\cdot\bar{a})=\prod_{i=1}^{p-1}\bar{i}\] By the definition of $Z_{p}\backslash\{\bar{0}\}$, $Z_{p}\backslash\{\bar{0}\}$ is communicative. So \[(\bar{a})^{p-1}(\prod_{i=1}^{p-1}\bar{i})=\prod_{i=1}^{p-1}\bar{i}\Rightarrow(\bar{a})^{p-1}=\bar{1}\]
\end{answer}

$$ $$

\begin{ex}
    Prove that there are only two distinct groups of order 4 (up to isomorphism), namely $Z_{4}$ and $Z_{2}\oplus Z_{2}$.
\end{ex}

\begin{answer}
    The only cyclic group of order 4 is $Z_{4}$. For a group $G$ of order 4 which is not cyclic, $\forall a\in G$, $a\neq e$, if $\left| a \right| =2$, $G\cong Z_{2}\oplus Z_{2}$. If there exists $a\in G$, $\left| a \right| =4$, $G\cong Z_{4}$. If there exists $a\in G$, $\left| a \right| =3$, denote $a^{2}=b, a^{3}=e$. Then $b^{2}=a^{4}=a$, $\{e,a,b\}<G$, which is contradictory to the Largrange theorem.
\end{answer}

$$ $$

\begin{ex}
    Let $H,K$ be subgroups of a group $G$. Then $HK$ is a subgroup of $G$ if and only if $HK=KH$.
\end{ex}

\begin{answer}
    If $HK=KH$, for $a_{1}b_{1},a_{2}b_{2}\in HK$, \[(a_{1}b_{1})(a_{2}b_{2})^{-1}=(a_{1}b_{1})(b_{2}^{-1}a_{2}^{-1})=(a_{1}b_{1})(a_{3}b_{3})\] since $b_{2}^{-1}a_{2}^{-1}\in KH=HK$, there exists $b_{2}^{-1}a_{2}^{-1}=a_{3}b_{3}$. \[(a_{1}b_{1})(a_{3}b_{3})=a_{1}(b_{1}a_{3})b_{3}=a_{1}a_{4}b_{4}b_{3}\] since $b_{1}a_{3}\in KH=HK$, there exists $b_{1}a_{3}=a_{4}b_{4}$. $(a_{1}b_{1})(a_{2}b_{2})^{-1}=a_{1}a_{4}b_{4}b_{3}=a_{5}b_{5}\in HK$. Thus $HK$ is a subgroup of $G$.

    If $HK$ is a subgroup of $G$, $\forall b_{1}a_{1}\in KH$, there exists $(a_{1}^{-1}b_{1}^{-1})\in HK$ s.t. $b_{1}a_{1}=(a_{1}^{-1}b_{1}^{-1})^{-1}\in HK$. So $KH\subset HK$. $\forall a_{1}b_{1}\in HK$, $(a_{1}b_{1})^{-1}=b_{1}^{-1}a_{1}^{-1}\in HK$, so $\exists a_{2}b_{2}\in HK$ s.t. $b_{1}^{-1}a_{1}^{-1}=a_{2}b_{2}$. $a_{1}b_{1}=b_{2}^{-1}a_{2}^{-1}\in KH$. So $HK\subset KH$. Thus $HK=KH$.
\end{answer}

$$ $$

\begin{ex}
    Let $G$ be a group of order $p^{k}m$, with $p$ prime and $(p,m)=1$. Let $H$ be a subgroup of order $p^{k}$ and $K$ a subgroup of order $p^{d}$, with $0<d\leq k$ and $K\not\subset H$. Show that $HK$ is not a subgroup of $G$.
\end{ex}

\begin{answer}
    Assume $HK<G$, $\left| HK \right| =p^{k}n$, $n|m$. We can get $\left[HK:H\right]=n=\left[K:K\cap H\right]$. $\left[K:K\cap H\right] | p^{k}\Rightarrow n|p^{k}$. That's contradictory to $(m,p^{k})=1$.
\end{answer}

$$ $$

\begin{ex}
    If $H$ and $K$ are subgroups of finite index of a group $G$ such that $\left[G:H\right]$ and $\left[G:K\right]$ are relatively prime, then $G=HK$.
\end{ex}

\begin{answer}
    Assume $\left[G:H\right]=m$, $\left[G:K\right]=n$, $(m,n)=1$. Then $\left| H \right| =np$, $\left| K \right|=mp$. $H\cap K<H$, $H\cap K<G\Rightarrow\left| H\cap K \right| |p$. \[\left[G:H\right]=m\geq \left[K:H\cap K\right]=\frac{\left| K \right| }{\left| H\cap K \right| }\geq m\] Thus $\left[G:H\right]=\left[K:H\cap K\right]=m$, $G=HK$.
\end{answer}

$$ $$

\begin{ex}
    If $H,K$ and $N$ are subgroups of a group $G$ such that $H<N$, then $HK\cap N=H(K\cap N)$. 
\end{ex}

\begin{answer}
    $\forall x=hk\in HK\cap N$, $\exists h_{1}^{-1}\in H$ s.t. $h_{1}^{-1}hk\in K\cap N$. $H<N$ so $\forall h_{1}^{-1}\in H, h_{1}^{-1}hk\in N$. Take $h_{1}^{-1}=h^{-1}$, $h_{1}^{-1}hk=k\in K$. So $HK\cap N\subset H(K\cap N)$.

    $\forall x= hk\in H(K\cap N)$ where $h\in H$, $k\in K\cap N$. $hk\in HK, h,k\in N\Rightarrow hk\in N$. So $H(K\cap N)\subset HK\cap N$. 

    Thus, $HK\cap N=H(K\cap N)$.
\end{answer}

$$ $$

\begin{ex}
    Let $H,K,N$ be subgroups of a group $G$ such that $H<K$, $H\cap N=K\cap N$, and $HN=KN$. Show that $H=K$.
\end{ex}

\begin{answer}
    Assume there exists $x\in K\backslash H$. $K\bigcup\limits_{i\in I}Ha_{i}$, $\forall h_{i}\in H$ there exists $a\in K$ s.t. $x=h_{1}a$. Take $n_{1}\in N$. Since $HN=KN$, $xn_{1}\in HN$, there exists $h_{2}\in H$, $n_{2}\in N$ s.t. $xn_{1}=h_{2}n_{2}=h_{2}an_{1}$. So $a=n_{2}n_{1}^{-1}\in N$, $a\in K\cap N=H\cap N\Rightarrow a\in H, x\in H$. That's contradictory!
\end{answer}

$$ $$

\begin{ex}
    Let $G$ be a group of order $2n$; then $G$ contains an element of order 2. If $n$ is odd and $G$ abelian, there is only one element of order 2.
\end{ex}

\begin{answer}
    The proof of the first part is exactly the same as \textbf{Exercise 1.1.14}.

    Assume there exists $a, b\in G$, $a^{2}=b^{2}=e$. We can check $H=\{e,a,b,ab\}$ is a subgroup of $G$. $\left| H \right| | \left| G \right| \Rightarrow 4|2n\Rightarrow 2|n$, which is contradictory to $n$ is odd. So there's only one element $a$ s.t. $a^{2}=e$.
\end{answer}

$$ $$

\begin{ex}
    If $H$ and $K$ are subgroups of a group $G$, then $\left[H\vee K:H\right]\\\geq \left[K:H\cap K\right]$.
\end{ex}

\begin{answer}
    The question is a direct corollary of Proposition 4.8.
\end{answer}

$$ $$

\begin{ex}
    If $p>q$ are primes, a group of order $pq$ has at most one subgroup of order $p$.
\end{ex}

\begin{answer}
    $H\cap K<H$, $H\cap K<K$, $H\neq K\neq H\cap K$. $\left| H\cap K \right| |p$ and $\left| H\cap K \right| \neq q$, so $H\cap K=\{e\}$. From \textbf{Exercise 1.3.12}, \[\left[H\vee K:H\right]\geq \left[K:K\cap H\right]=p\]\[\left| H\vee K \right| =\left| H \right|\cdot \left[H\vee K:H\right]\geq p^{2}\] But $H\vee K\in G$, $\left| H\vee K \right| \leq pq<p^{2}$. That's contradictory!
\end{answer}

$$ $$

\begin{ex}
    Let $G$ be a group and $a,b\in G$ such that (i) $\left| a \right| =4=\left| b \right| $; (ii) $a^{2}=b^{2}$; (iii) $ba=a^{3}b=a^{-1}b$; (iv) $a\neq b$; (v) $G=\left\langle a,b\right\rangle$. Show that $\left| G \right| =8$ and $G\cong Q_{8}$.
\end{ex}

\begin{answer}
    The proof is exactly the same as \textbf{Exercise 1.2.3}.
\end{answer}