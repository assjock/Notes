\section{Cyclic groups}
\begin{ex}
    Let $a,b$ be elements of group $G$. Show that $\left| a \right| =\left| a^{-1} \right| $; $\left| ab \right| =\left| ba \right| $, and $\left| a \right| =\left| cac^{-1} \right| $ for all $c\in G$.
\end{ex}

\begin{answer}
    We only consider that $\left| a \right| , \left| b \right| , \left| c \right| $ are finite. Assume $a^{k}=e$, $(ab)^{m}=e$, $(ac^{-1})^{n}=e$, $kmn\neq 0$. $a^{k}\cdot(a^{-1})^{k}=e$, so $k$ sialso the order of $a^{-1}$, $\left| a^{-1} \right| =k$. $(ab)^{m}=e=a(ba)^{m-1}b\Rightarrow (ba)^{m-1}=a^{-1}b^{-1}$, $(ba)^{m}=a^{-1}b^{-1}ba=e$. $m$ is the order of $ba$. $(cac^{-1})^{r}=cac^{-1}cac^{-1}\cdots cac^{-1}=ca^{n}c^{-1}=e$, so $a^{n}=e$, whence $n=k$.
\end{answer}

$$ $$

\begin{ex}
    Let $G$ be an abelian group containing elements $a$ and $b$ of orders $m$ and $n$ respectively. Show that $G$ contains an element whose order is the least commom multiple of $m$ and $n$.
\end{ex}

\begin{answer}
    If $(m,n)=1$, we know that $\forall a^{i}, i=1,2,\dots, m$, $b^{j}, j=1, 2, \dots, n$, $a^{i}b^{j}\neq e$, since if $a^{i}=b^{j}$, $\left| a^{i} \right| =n=\left| b^{-j} \right| =\left| b^{j} \right| =m$. $G$ is abelian, so $(ab)^{k}=a^{k}b^{k}\Rightarrow \left| ab \right|=mn=\left[ m,n\right]$.

    If $m|n$ or $n|m$, then $a$ or $b$ is the element we want. We consider $m|\!\!/n$ and $n|\!\!/m$. Factorise $n=p_{1}^{t_{1}}p_{2}^{t_{2}}\cdots p_{l}^{t_{l}}$, $m=p_{1}^{s_{1}}p_{2}^{s_{2}}\cdots p_{l}^{s_{l}}$, where $p_{1},\cdots,p_{l}$ are primes and $t_{1},\cdots,t_{l}, s_{1},\cdots, s_{l}\geq 0$. We can choose a new arrangement of $p_{1},\cdots,p_{l}$ and make $t_{1}\geq s_{1}$, $t_{2}\geq s_{2}$,\dots, $t_{i}\geq s_{i}$, $t_{i+1}<s_{i+1}$,\dots, $t_{l}<s_{l}$.\[(m,n)=p_{1}^{s_{1}}\cdots p_{i}^{s_{i}}p_{i+1}^{t_{i+1}}\cdots p_{l}^{t_{l}}, \left[m,n\right]=p_{1}^{t_{1}}\cdots p_{i}^{t_{i}}p_{i+1}^{s_{i+1}}\cdots p_{l}^{s_{l}}\] Take $x=a^{{p_{i+1}^{s_{i+1}}}\cdots p_{l}^{s_{l}}}$, $y=b^{{p_{1}^{t_{1}}}\cdots p_{i}^{t_{i}}}$, then $\left| x \right| ={p_{1}^{t_{1}}}\cdots p_{i}^{t_{i}}$, $\left| y \right| =p_{i+1}^{s_{i+1}}\cdots p_{l}^{s_{l}}$. Thus $(x,y)=1$, the order of $xy$ is $\left| x \right| \cdot\left| y \right| =p_{1}^{t_{1}}\cdots p_{i}^{t_{i}}p_{i+1}^{s_{i+1}}\cdots p_{l}^{s_{l}}=\left[m,n\right]$.
\end{answer}

$$ $$

\begin{ex}
    Let $G$ be an abelian group of order $pq$, with $(p,q)=1$. Assume there exist $a,b\in G$ such that $\left| a \right| =p, \left| b \right| =q$ and show that $G$ is cyclic.
\end{ex}

\begin{answer}
    From \textbf{Exercise 1.3.2} we know $a^{i}b^{j}\neq e$ for $i<p$, $j<q$. $\left| G \right| =pq$ for all $a^{i}b^{j}$ and $a^{m}b^{n}$ with $i\neq m$, $b\neq n$, $a^{i}b^{j}\neq a^{m}b^{n}$. So $G$ can be generated by $ab$. $G$ is cyclic.
\end{answer}

$$ $$

\begin{ex}
    If $f:G\to H$ is a homomorphism, $a\in G$, and $f(a)$ has finte order in $H$, then $\left| a \right| $ is infinite or $\left| f(a) \right| $ divides $\left| a \right| $.
\end{ex}

\begin{answer}
    Assume $\left| f(a) \right| =n$, $\left| a \right| =m$, and $n|\!\!/m$. Trivially, $m\geq n$. Assume $\gcd(m,n)=k\leq n$. $a^{m}=e\Rightarrow f(a)^{m}=e'=f(a)^{n}$. By Bezout theorem $\exists x,y\in \mathbf{Z}$ s.t. $f(a)^{mx+ny}=f(a)^{k}=e'$, $k\leq n$, that's contradictory!
\end{answer}

$$ $$

\begin{ex}
    Let $G$ be the multiplicative group of all nonsingular $2\times 2$ matrices with rational entries. Show that $a=\begin{pmatrix}
        0 & -1\\1 & 0
    \end{pmatrix}$has order 4 and $b=\begin{pmatrix}
        0& 1\\-1&-1
    \end{pmatrix}$has order 3, but $ab$ has infinite order. Conversely, show that the additive group $Z_{2}\oplus \mathbf{Z}$ contains nonzero elements $a,b$ of infinite order such that $a+b$ has finite order. 
\end{ex}

\begin{answer}
    The verification of $\left| a \right| =4$ and $\left| b \right| =3$ is trivial. $ab=\begin{pmatrix}
        1 & 1\\ 0& 1
    \end{pmatrix}$ $\det(ab=\lambda I)=0\Rightarrow \lambda_{1}=\lambda_{2}=1$. $ab$ is not diagnizable. By induction, we have $(ab)^{n}=\begin{pmatrix}
        1 & 2^{n-1}\\ 0& 1
    \end{pmatrix}$ which means $(ab)$ has infinite order.

    For $a=(\bar{0},1), b=(\bar{0},-1)\in Z_{2}\oplus\mathbf{Z}$, $a,b$ have infinite order, but $a+b=(\bar{0},0)$ has finite order 1.
\end{answer}

$$ $$

\begin{ex}
    If $G$ is a cyclic group of order $n$ and $k| n$, then $G$ has exactly one subgroup of order $k$.
\end{ex}

\begin{answer}
    Assume $a^{n}=e$, $mk=n$, we verify that $\left\langle a^{m}\right\rangle$ is a subgroup of order $k$. $\forall x,y\in \mathbf{Z}_{+}$, $a^{xm}\cdot a^{-ym}=a^{(x-y)m}\in \left\langle a^{m}\right\rangle$, so $\left\langle a^{m}\right\rangle$ is a subgroup. $a^{km}=e$, $a^{sm}\neq e$ for $s<k$, so $\left| \left\langle a^{m}\right\rangle \right| =k$.
\end{answer}

$$ $$

\begin{ex}
    Let $p$ be prime and $H$ a subgroup of $Z(p^{\infty})$.
    \begin{enumerate}[(a)]
        \item Every element of $Z(p^{\infty})$ has finite order $p^{n}$ for some $n\geq 0$.
        \item If at least one element of $H$ has order $p^{k}$ and no element of $H$ has order greater than $p^{k}$, then $H$ is the cyclic subgroup generated by $\bar{1/p^{k}}$, whence $H\cong Z_{p^{k}}$.
        \item If there is no upper bound on the orders of elements of $H$, then $H=Z(p^{\infty})$.
        \item The only proper subgroups of $Z(p^{\infty})$ are the finite cyclic groups $C_{n}=\left\langle\bar{1/p^{n}}\right\rangle\,(n=1,2,\dots)$. Futhermore, $\left\langle0\right\rangle=C_{0}<C_{1}<C_{2}<C_{3}<\cdots$.
        \item Let $x_{1},x_{2},\dots$ be elements of an abelian group $G$ such that $\left| x_{1} \right| =p, px_{2}=x_{1},px_{3}=x_{2},\dots,px_{n+1}=x_{n},\dots$. The subgroup generated by the $x_{i}(i\geq 1)$ is isomorphic to $Z(p^{\infty})$. 
    \end{enumerate}
\end{ex}

\begin{answer}
    \begin{enumerate}[(a)]
        \item $\forall  x\in Z(p^{\infty})$, $x=\frac{a}{p^{n}}$ where $a<p^{n}$, $p|\!\!/a$. $p$ is a prime, so $\gcd(p,a)=1$. $m\cdot a|p^{n}\Rightarrow m=p^{n}$. Thus $m\cdot \frac{a}{p^{n}}=e$, $p^{n}$ is the smallest number satisfies it. $\frac{a}{p^{n}}$ has order $p^{n}$.
        \item For all $x\in Z(p^{\infty})$, if $x$ has order smaller than $p^{k}$, $x$ must have the form $x=\frac{a}{p^{i}}(i\leq k)$, $(p,a)=1$, so $x\in\left\langle\frac{1}{p^{k}}\right\rangle$. If not, assume $x=\frac{a}{p^{i}}(i>k)$, then $p^{k}\cdot x=\frac{a}{p^{i-k}}\neq 1$.
        \item Assume not, $H< Z(p^{\infty})$, $H\neq Z(p^{\infty})$. There exist $y\in H$ s.t. $y$ has order $p^{m}, m\geq n$. $y=\frac{b}{p^{m}}$, $(p,b)=1$, so there exists $b^{-1}\in\{1,2,\dots,p-1\}$, $bb^{-1}\equiv 1\mod p^{m}$. But $ab^{-1}p^{m-n}y=\frac{a}{p^{n}}=x\in H$, that's contradictory! Conversely, $H=Z(p^{\infty})$.
        \item From (b), we know that if there's least upper bound $p^{n}$ for elements in a subgroup $S$, then $S=C_{n}$.\[\left\langle0\right\rangle=C_{0}<C_{1}<C_{2}<C_{3}<\cdots<Z(p^{\infty})\] is easy to verify.
        \item We can verify that $f:x_{i}\mapsto \bar{\frac{1}{p^{i}}}$ is a well defined isomorphism. $f(e)=f(px_{1})=1$, $f(px_{i+1})=f(x_{i})=\frac{1}{p^{i}}=p\cdot \frac{1}{p^{i+1}}$. $f$ is obviously a bijection, so $H\cong Z(p^{\infty})$.
    \end{enumerate}
\end{answer}

$$ $$

\begin{ex}
    A group that has only a finite number of subgroups must be finite.
\end{ex}

\begin{answer}
    Suppose not. If the order of all subgroups are finite, $G$ must be finite. So there exists a infinite subgroup $H<G$. $\forall  a\in G$, if $\forall n\in \mathbf{N}$, $a^{n}\neq e$. then we can construct infinte subgroups $\left\langle a\right\rangle$, $\left\langle a^{2}\right\rangle$, $\left\langle a^{3}\right\rangle$\dots. If $\forall a\in G$, $\exists n\in \mathbf{N}$, $a^{n}=e$, so $\left\langle a\right\rangle$ is a proper subgroup of $G$, we can take $b\in G\backepsilon\left\langle a\right\rangle$ to construct another subgroup. By induction, there are infinte subgroups in $G$. That's contradictory, so $G$ must be finite.
\end{answer}

$$ $$

\begin{ex}
    If $G$ is an abelian group, then the set $T$ of all elements of $G$ with finite order is a subgroup of $G$.
\end{ex}

\begin{answer}
    We can easily verify that $\forall a,b\in S$, $\left| a \right| =m$, $\left| b \right| =n$ and $\left| ab^{-1} \right| \leq mn$ is finite. $T$ is a subgroup of $G$.
\end{answer}

$$ $$

\begin{ex}
    An infinite group is cyclic if and only if it is isomorphic to each of its proper subgroups.
\end{ex}

\begin{answer}
    If $G$ is cyclic, $G\cong \mathbf{Z}$, $S<G$. For any subgroup of $\mathbf{Z}$, it has the form $\{na\}, a\in \mathbf{Z}$. We can construct a isomorphism $f:n\mapsto na$, so $S\cong \{na\}\Rightarrow G\cong S$.

    If $\forall S<G$, $G\cong S$ and $\left| G \right| =\left| S \right| $ is finite. We prove there exists $S<G$ s.t. $\left| S \right| =\aleph_{0}$. Take $a\in G$ and $S=\{na|n\in \mathbf{Z}\}$, $S$ is a subgroup. If there exists $ma=0$, $S$ must be finite, contradictory! Thus, $S\cong \mathbf{Z}\cong G$. $G$ is a infinite cyclic group.
\end{answer}