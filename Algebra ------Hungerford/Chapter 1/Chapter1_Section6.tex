\section{Symmetric, alternating, and dihedral groups}
\begin{ex}
    Find four different subgroups of $S_{4}$ that are isomorphic to $S_{3}$ and nine isomorphic to $S_{2}$.
\end{ex}

$$ $$

\begin{ex}
    \begin{enumerate}[(a)]
        \item $S_{n}$ is generated by the $n-1$ transpositions $(12)$, $(13)$, $(14)$, $\dots$, $(1n)$.
        \item $S_{n}$ is generated by the $n-1$ transpositions $(12), (23), (34),\dots, (n-1\, n)$.
    \end{enumerate}
\end{ex}

$$ $$

\begin{ex}
    If $\sigma=(i_{1}i_{2}\cdots i_{r})\in S_{n}$ and $\tau\in S_{n}$, then $\tau\sigma\tau^{-1}$ is the $r$-cycle $(\tau(i_{1})\tau(i_{2})\cdots\tau(i_{r}))$.
\end{ex}

$$ $$

\begin{ex}
    \begin{enumerate}[(a)]
    \item $S_{n}$  is generated by $\sigma_{1}=(12)$ and $\tau=(123\cdots n)$.
    \item $S_{n}$ is generated by $(12)$ and $(23\cdots n)$.
    \end{enumerate}
\end{ex}

$$ $$

\begin{ex}
    Let $\sigma, \tau \in S_{n}$. If $\sigma$ is even (odd), then so is $\tau\sigma\tau^{-1}$.
\end{ex}

$$ $$

\begin{ex}
    $A_{n}$ is the only subgroup of $S_{n}$ of index 2.
\end{ex}

$$ $$

\begin{ex}
    Show that $N=\{(1),(12)(34),(13)(24),(14)(23)\}$ is a normal subgroup of $S_{4}$ contained in $A_{4}$ such that $S_{4} /N\cong S_{3}$ and $A_{4} /N\cong Z_{3}$.
\end{ex}

$$ $$

\begin{ex}
    The group $A_{4}$ has no subgroup of order $6$.
\end{ex}

$$ $$

\begin{ex}
    For $n\geq 3$ let $G_{n}$ be the multiplicaive group of complex matrices generated by $x=\begin{pmatrix}
        0&1\\1&0
    \end{pmatrix}$ and $y=\begin{pmatrix}
        e^{2\pi i/n}&0\\0&e^{-2\pi i/n}
    \end{pmatrix}$, where $i^{2}=-1$. Show that $G_{n}\cong D_{n}$.
\end{ex}

$$ $$

\begin{ex}
    Let $a$ be the generator of order $n$ of $D_{n}$. Show that $\left\langle a\right\rangle\lhd D_{n}$ and $D_{n} /\left\langle a\right\rangle\cong Z_{2}$.
\end{ex}

$$ $$

\begin{ex}
    Find all normal subgroups of $D_{n}$.
\end{ex}

$$ $$

\begin{ex}
    The center of the group $D_{n}$ is $\left\langle e\right\rangle$ if $n$ is odd and isomorphic to $Z_{2}$ if $n$ is even.
\end{ex}

$$ $$

\begin{ex}
    For each $n\geq 3$ let $P_{n}$ be a regular polygon of $n$ sides (for $n=3$, $P_{n}$ is an equilateral triangle; for $n=4$, a square). A \emph{symmetry} of $P_{n}$ is a bijection $P_{n}\to P_{n}$ that preserves distances and maps adjacent vertices on to adjacent vertices.
    
    \begin{enumerate}[(a)]
        \item The set $D_{n}^{*}$ of all symmetries of $P_{n}$ is a group under the binary operation of composition of functions.
        \item Every $f\in D_{n}^{*}$ is completely determined by its actions on the vertices of $P_{n}$. Number the vertices consecutively $1,2,\dots, n$; then each $f\in D_{n}^{*}$ determines a unique permutation $\sigma_{f}$ of $\{1,2,\dots, n\}$. The assignment $f\mapsto \sigma_{f}$ defines a monomorphism of groups $\varphi:D_{n}^{*}\to S_{n}$.
        \item $D_{n}^{*}$ is generated by $f$ and $g$, where $f$ is a rotation of $2\pi /n$ degrees about the center of $P_{n}$ and $g$ is a reflection about the ``diameter'' through the center and vertex 1.
        \item $\sigma_{f}=(123\cdots n)$ and $\sigma_{g}=\begin{pmatrix}
            1&2&3&\cdots&n-1&n\\1&n&n-1&\cdots&3&2
        \end{pmatrix}$, whence \\$\mathrm{Im}\varphi=D_{n}$ and $D_{n}^{*}\cong D_{n}$.
    \end{enumerate}
\end{ex}