\section{Semigroups, monoids and groups}
\begin{ex}
    Give examples other than those in the text of semigroups and monoids that are not groups.
\end{ex}

$$ $$

\begin{ex}
    Let $G$ be a group (written additively), $S$ a nonempty set, and $M(S,G)$ the set of all functions $f: S \rightarrow G$. Define addition in $M(S,G)$ as follows: $(f + g) : S \rightarrow G$ is given by $s \rightarrow f(s) + g(s) \in G$. Prove that $M(S,G)$ is a group, which is abelian if $G$ is.
\end{ex}

$$ $$

\begin{ex}
    Is it true that a semigroup which has a left identity element and in which every element has a right inverse (see Proposition 1.3) is a group?
\end{ex}

$$ $$

\begin{ex}
    Write ou a multiplication table for the group $D_4^*$.
\end{ex}

$$ $$

\begin{ex}
    Prove that the symmetric group on $n$ letters, $S_n$, has ordrer $n!$.
\end{ex}

$$ $$

\begin{ex}
    Write out an addition table for $Z_2\oplus Z_2$. $Z_2\oplus Z_2$ is called the Klein four group.
\end{ex}

$$ $$

\begin{ex}
    If $p$ is prime, then the nonzero elements of $Z_p$ form a group of order $p - 1$ under multiplication. [Hint: $\bar{a} \neq 0 \Rightarrow (a,p) = 1$ ; use Introduction, Theorem 6.5.] Show that this statement is false if $p$ is not prime.
\end{ex}

$$ $$

\begin{ex}
    \begin{enumerate}
        \item The relation given by $a ~ b \Leftrightarrow a - b \in \mathbf{Z}$ is a congruence relation on the additive group $\mathbf{Q}$ [see Theorem 1.5].
        \item The set $\mathbf{Q}/\mathbf{Z}$ of equivalence classes is an infinite abelian group.
    \end{enumerate}
\end{ex}

$$ $$

\begin{ex}
    Let $p$ be a fixed prime. Let $R_p$ be the set of all those rational numbers whose de nominator is relatively prime to $p$. Let $R^p$ be the set of rationals whose de nominator is a power of $p (p^i, i > 0)$. Prove that both $R_p$ and $R^p$ are abelian groups under ordinary addition of rationals.
\end{ex}

$$ $$

\begin{ex}
    Let $p$ be a prime and let $Z(p^\infty)$ be the following subset of the group $\mathbf{Q}/\mathbf{Z}$:\[Z(p^\infty)=\{\bar{a/b}\in\mathbf{Q}/\mathbf{Z}| a,b \in \mathbf{Z} \text{ and } b=p^i \text{ for some }i\geq 0\}\]
    Show that $Z(p^\infty)$ is an infinite group under the addition operation of $\mathbf{Q}/\mathbf{Z}$.
\end{ex}

$$ $$

\begin{ex}
    The following conditions on a group $G$ are equivalent:
    \begin{enumerate}
        \item $G$ is abelian,;
        \item $(ab)^2=a^{2}b^{2}$ for all $a,b\in G$;
        \item $(ab)^{-1}=a^{-1}b^{-1}$ for all $a,b \in G$;
        \item $(ab)^{n}=a^{n}b^{n}$ for all $n\in \mathbf{Z}$ and all $a,b \in G$;
        \item $(ab)^{n}=a^{n}b^{n}$ for three consecutive integers $n$ and all $a,b \in G$. Show that $(v)\Rightarrow (i)$ is false if `three' is replaced by `two'.
    \end{enumerate}
\end{ex}

$$ $$

\begin{ex}
    If $G$ is a group, $a,b\in G$ and $bab^{-1}=a^{r}$ for some $r\in \mathbf{N}$, then $b^{j}ab^{-j}=a^{r^{j}}$ for all $j\in \mathbf{n}$. 
\end{ex}

$$ $$

\begin{ex}
    If $a^{2}=e$ for all elements $a$ of a group $G$, then $G$ is abelian.
\end{ex}

$$ $$

\begin{ex}
    If $G$ is a finite group of even order, then $G$ contains an element $a\neq e$ such that $a^{2}=e$.
\end{ex}

$$ $$

\begin{ex}
    Let $G$ be a nonempty finite set with an associative binary operation such that for all $a, b, c\in G$, $ab = ac\Rightarrow b = c$ and $ba  =ca \Rightarrow b = c$. Then $G$ is a group. Show that this conclusion may be false if $G$ is finite.
\end{ex}

$$ $$

\begin{ex}
    Let $a_1, a_2,\dots$ be a sequence of elements in a semigroup $G$. Then there exists a unique function $\Psi: \mathbf{N^*}\rightarrow G$ such that $\Psi(1)=a_1, \Psi(2)=a_1a_2, \Psi(3)=(a_1a_2)a_3$ and for $n\geq 1$, $\Psi(n+1)=(\Psi(n))a_{n+1}$. Note that $\Psi(n)$ is precisely the standard $n$ product $\prod_{i=1}^na_i$.
\end{ex}