\section{Semigroups, monoids and groups}
\begin{ex}
    Give examples other than those in the text of semigroups and monoids that are not groups.
\end{ex}

\begin{answer}
    Semigroup: $(\mathbf{Z}_+, +)$

    Monoid: $(\mathbf{Z}_+, \times )$ 
\end{answer}

$$ $$

\begin{ex}
    Let $G$ be a group (written additively), $S$ a nonempty set, and $M(S,G)$ the set of all functions $f: S \rightarrow G$. Define addition in $M(S,G)$ as follows: $(f + g) : S \rightarrow G$ is given by $s \rightarrow f(s) + g(s) \in G$. Prove that $M(S,G)$ is a group, which is abelian if $G$ is.
\end{ex}

\begin{answer}
    Firstly we check $M(S,G)$ is a group
    \begin{enumerate}
        \item $f+g: s\mapsto f(s) + g(s) \in G$, so $f+g\in M(S,G)$
        \item $(f+g)+h: s\mapsto (f(s) + g(s)) + h(s)$, $G$ is a group, so $s\mapsto (f(s) + g(s)) + h(s)\Leftrightarrow s\mapsto f(s) + (g(s) + h(s))$, $(f+g)+h = f+(g+h)$.
        \item Take the unit element as $e': s\mapsto e$. $f+e': s\mapsto f(s)+ e'(s) =f(s)+e=f(s)$, so $f+e'=f$. Similarly, $e'+f = f$.
        \item For any $f\in M(S,G)$, take $f^{-1}: s\mapsto (f(s))^{-1}$, whence $f(s)+(f(s))^{-1}=(f(s))^{-1}+f(s)=e$.
    \end{enumerate}
    In conclusion, $M(S,G)$ is a group. If $G$ is abelian $f+g: s\mapsto f(s)+g(s)=g(s)+f(s)$, $f+g=g+f$, so $M(S,G)$ is abelian.
\end{answer}

$$ $$

\begin{ex}
    Is it true that a semigroup which has a left identity element and in which every element has a right inverse (see Proposition 1.3) is a group?
\end{ex}

\begin{answer}
    If $e$ is the left identity, $\forall a \in A, ea=a$ and $\forall a\in A ,\exists  a^{-1} s.t. aa^{-1}=e$. We have proved that if $cc=c$, then $c=e$. \[(a^{-1}a)(a^{-1}a)=a^{-1}(aa^{-1})a=a^{-1}(ea)=a^{-1}a\Rightarrow a^{-1}a=e\]
    $a^{-1}$ is also the left inverse. $ae=a(a^{-1}a)=(aa^{-1})a=ea=a$, $e$ is also the right identity.
\end{answer}

$$ $$

\begin{ex}
    Write out a multiplication table for the group $D_4^*$.
\end{ex}

\begin{answer}
    $D_4^*=\{R,R^{2}, R^{3},I,T_x,T_y,T_{13},T_{24}\}$
    \begin{table}[H]
        \centering
        \begin{tabular}{c|cccccccc}
        \multicolumn{1}{c}{} & $I$ & $R$ & $R^{2}$ & $R^{3}$ & $T_x$ & $T_y$ & $T_{13}$ & $T_{24}$  \\ 
        \cline{2-9}
        $I$ & $I$ & $R$ & $R^{2}$ & $R^{3}$ & $T_x$ & $T_y$ & $T_{13}$ &  $T_{24}$ \\
        $R$ & $R$ & $R^{2}$ & $R^{3}$ & $I$ & $T_{13}$ & $T_{24}$ & $T_y$ & $T_x$  \\
        $R^{2}$ & $R^{2}$ & $R^{3}$ & $I$ & $R$  & $T_y$ & $T_x$ & $T_{24}$ & $T_{13}$  \\
        $R^{3}$ & $R^{3}$ & $I$ & $R$  & $R^{2}$ & $T_{24}$ & $T_{13}$ & $T_x$ & $T_y$  \\
        $T_x$ & $T_x$ & $T_{24}$ & $T_y$ & $T_{13}$ & $I$ & $R^{2}$ & $R^{3}$ & $R$  \\
        $T_y$ & $T_y$ & $T_{13}$ & $T_x$ & $T_{24}$ & $R^{2}$ & $I$ & $R$ & $R^{3}$  \\
        $T_{13}$ & $T_{13}$ & $T_y$ & $T_{24}$ & $T_x$ & $R^{3}$ & $R$ & $I$ &  $R^{2}$ \\
        $T_{24}$ & $T_{24}$ & $T_x$ & $T_{13}$ & $T_y$ & $R$ & $R^{3}$ & $R^{2}$ &  $I$
        \end{tabular}
        \end{table}
\end{answer}

$$ $$

\begin{ex}
    Prove that the symmetric group on $n$ letters, $S_n$, has ordrer $n!$.
\end{ex}

\begin{answer}
    For a set $A$ whose order is $n$, we prove there's $n!$ different bijections by induction
    \begin{enumerate}
        \item For $n=1$, trivial.
        \item Assume $n=k$, there's $k!$ bijections. For $n=k+1$, fix one element in $A$, and take $a\mapsto a$, there's $k$ free elements, so there's $k!\cdot(k+1)$ bijections in total.
    \end{enumerate}
    By induction, we get the result.
\end{answer}

$$ $$

\begin{ex}
    Write out an addition table for $Z_2\oplus Z_2$. $Z_2\oplus Z_2$ is called the Klein four group.
\end{ex}

\begin{answer}
    $Z_2=\{1,0\}$, $Z_2\oplus Z_2=\{(1,1), (1,0), (0,1), (0,0)\}$
    \begin{table}[H]
        \centering
        \begin{tabular}{c|cccc}
            \multicolumn{1}{c}{} & $(1,1)$ & $(1,0)$ & $(0,1)$ & $(0,0)$\\
            \cline{2-5}
            $(1,1)$ & $(0,0)$ & $(0,1)$ & $(1,0)$ & $(1,1)$ \\
            $(1,0)$ &$(0,1)$  & $(0,0)$ & $(1,1)$ & $(1,0)$ \\
            $(0,1)$ & $(1,0)$ & $(1,1)$ & $(0,0)$ & $(0,1)$ \\
            $(0,0)$ & $(1,1)$ & $(1,0)$ & $(0,1)$ & $(0,0)$
        \end{tabular}
    \end{table}
\end{answer}

$$ $$

\begin{ex}
    If $p$ is prime, then the nonzero elements of $Z_p$ form a group of order $p - 1$ under multiplication. Show that this statement is false if $p$ is not prime.
\end{ex}

\begin{answer}
    For the set $Z_p\backslash\{\bar{0}\}$
    \begin{enumerate}
        \item $Z_p\backslash\{\bar{0}\}$ is obviously associative and communicative.
        \item Take $\bar{1}$ as the identity element, $\forall \bar{a}\in Z_p\backslash\{\bar{0}\}, \bar{1}\times \bar{a}=\bar{a}$.
        \item We prove there is a unique element $a^{-1}\in Z_p\backslash\{\bar{0}\} s.t. aa^{-1}=\bar{1} $. Assume there exists $\bar{b},\bar{c}$ and $\bar{a}\cdot\bar{b}=\bar{k},\bar{a}\cdot\bar{c}=\bar{k}$, then $a(b-c)\equiv 0\mod{p}$. $p$ is a prime, so $lcm(p,a)=1,  lcm(p,b-c)=1$, so $\bar{b}=\bar{c}$. There is at most one element s.t. $\bar{a}\bar{b}=\bar{k}$. Take $\bar{b}=\bar{1}, \bar{2},\dots\bar{p-1}$, $\bar{k}$ travels through $\bar{b}=\bar{1}, \bar{2},\dots\bar{p-1}$. There exists an element $\bar{b}\in Z_p\backslash\{\bar{0}\}, \bar{a}\bar{b}=\bar{1}$.
    \end{enumerate}
    $Z_p\backslash\{\bar{0}\}$ is a group. If $p$ is not a prime, the inverse element is not always unique. Take $a|p$, there's more than one inverse element in $Z_p\backslash\{\bar{0}\}$.
\end{answer}

$$ $$

\begin{ex}
    \begin{enumerate}[(a)]
        \item The relation given by $a \thicksim b \Leftrightarrow a - b \in \mathbf{Z}$ is a congruence relation on the additive group $\mathbf{Q}$ [see Theorem 1.5].
        \item The set $\mathbf{Q}/\mathbf{Z}$ of equivalence classes is an infinite abelian group.
    \end{enumerate}
\end{ex}

\begin{answer}
    \begin{enumerate}[(a)]
        \item For group $(\mathbf{Q}, +)$, $a_1\thicksim b_1\Leftrightarrow a_1-b_1=k_1\in \mathbf{Z}$, $a_2\thicksim b_2\Leftrightarrow a_2-b_2=k_2\in \mathbf{Z}$, so $(a_1+a_2)-(b_1+b_2)=((k_1+b_1)+(k_2+b_2))-(b_1+b_2)=k_1+k_2\in\mathbf{Z}$. $a\thicksim b$ is a congruence relation.
        \item \begin{enumerate}[1]
            \item if $a+b\geq 1$, $\bar{a}+\bar{b}=\bar{a+b-1}$. If $a+b<1$, $\bar{a}+\bar{b}=\bar{a+b}$.
            \item $\mathbf{Q}/\mathbf{Z}$ is obviously associative and communicative.
            \item Take the identity element as $\bar{0}$, $\bar{0}+\bar{a}=\bar{a}$.
            \item If $\bar{a}\neq 0$, take $(\bar{a})^{-1}=\bar{1-a}$, then $\bar{a}+\bar{1-a}=\bar{0}$
        \end{enumerate}
        so $\mathbf{Q}/\mathbf{Z}$ is a abelian group. (Infinite remains to be certified)
    \end{enumerate}
\end{answer}

$$ $$

\begin{ex}
    Let $p$ be a fixed prime. Let $R_p$ be the set of all those rational numbers whose denominator is relatively prime to $p$. Let $R^p$ be the set of rationals whose denominator is a power of $p (p^i, i > 0)$. Prove that both $R_p$ and $R^p$ are abelian groups under ordinary addition of rationals.
\end{ex}

\begin{answer}
    Trivial.
\end{answer}

$$ $$

\begin{ex}
    Let $p$ be a prime and let $Z(p^\infty)$ be the following subset of the group $\mathbf{Q}/\mathbf{Z}$:\[Z(p^\infty)=\{\bar{a/b}\in\mathbf{Q}/\mathbf{Z}| a,b \in \mathbf{Z} \text{ and } b=p^i \text{ for some }i\geq 0\}\]
    Show that $Z(p^\infty)$ is an infinite group under the addition operation of $\mathbf{Q}/\mathbf{Z}$.
\end{ex}

\begin{answer}
    $Z(p^\infty)=\{\bar{a/b}|a, b\in\mathbf{Z}, b=p^i, i\geq 0 \}$. Take $a=\bar{\frac{a_1}{b_1}}$, $b=\bar{\frac{a_2}{b_2}}$. $b^{-1}=\bar{\frac{b_2-a_2}{b_2}}$
    \[\begin{aligned}
        a+b^{-1}=\bar{\frac{a_1}{b_1}}+\bar{\frac{b_2-a_2}{b_2}}&=\bar{\frac{a_1}{p^{s_1}}}+\bar{\frac{p^{s_2}-a_2}{p^{s_2}}}\\ &=\bar{\frac{a_1\cdot p^{s_2}+p^{s_1}(p^{s_2}-a_2)}{p^{s_1+s_2}}}\in Z(p^\infty)
    \end{aligned}\]
    Therefore, $Z(p^\infty)$ is a subgroup of $\mathbf{Q}/\mathbf{Z}$. $\frac{1}{p^i}\in Z(p^\infty)$ for any $i \in \mathbf{Z}$, so $Z(p^\infty)$ is infinite, $\mathbf{Q}/\mathbf{Z}$ is also infinite.
\end{answer}

$$ $$

\begin{ex}
    The following conditions on a group $G$ are equivalent:
    \begin{enumerate}[i]
        \item $G$ is abelian;
        \item $(ab)^2=a^{2}b^{2}$ for all $a,b\in G$;
        \item $(ab)^{-1}=a^{-1}b^{-1}$ for all $a,b \in G$;
        \item $(ab)^{n}=a^{n}b^{n}$ for all $n\in \mathbf{Z}$ and all $a,b \in G$;
        \item $(ab)^{n}=a^{n}b^{n}$ for three consecutive integers $n$ and all $a,b \in G$. Show that v$\Rightarrow$ i is false if `three' is replaced by `two'.
    \end{enumerate}
\end{ex}

\begin{answer}
    i$\Leftrightarrow$ iii: $((ab)b^{-1})a^{-1}=(ab)(b^{-1}a^{-1})=e$, so $(ab)^{-1}=b^{-1}a^{-1}$. If iii, $b^{-1}a^{-1}=a^{-1}b^{-1}$ for any $a,b \in G$, $G$ is abelian. If i, $G$ is abelian, $(ab)^{-1}=b^{-1}a^{-1}=a^{-1}b^{-1}$.

    iv $\Rightarrow$ v, iv$\Rightarrow$ ii and i$\Rightarrow$ iv are trivial.

    ii$\Rightarrow$ i: \[(ab)(ab)=aabb\Rightarrow a^{-1}(ab)^{2}b^{-1}=a^{-1}aabbb^{-1}=ba=ab\] so $G$ is abelian.

    v $\Rightarrow$ i: $a^{n}b^{n}=(ab)^{n}$, $a^{n-1}b^{n-1}=(ab)^{n-1}$, $a^{n+1}b^{n+1}=(ab)^{n+1}$. \[(b^{-1})^{n}(a^{-1})^{n}=((ab)^{n})^{-1}=((ab)^{-1})^{n}\]\[((ab)^{-1})^{n}(ab)^{n+1}=(b^{-1})^{n}ab^{n+1}\]\[((ab)^{-1})^{n}(ab)^{n-1}=b^{-1}a^{-1}=(b^{-1})^{n}a^{-1}b^{n-1}\]\[a=(b^{-1})^{n}ab^{n}\qquad b^{-1}a^{-1}b=(b^{-1})^{n}a^{-1}b^{n}\]So $a^{-1}=b^{-1}a^{-1}b$, which means $G $ is abelian.

    If ``three'' is replaced by ``two'': $a^{n}b^{n}=(ab)^{n}$, $a^{n+1}b^{n+1}=(ab)^{n+1}$. \[(b^{-1})^{n}(a^{-1})^{n}=((ab)^{-1})^{n}\qquad a=(b^{-1})^{n}ab^{n}\] 
    For the group $S_3=\{(1),(12),(13),(23),(123),(132)\}$, taking any $a\in S_3$, we can check that $a^{6}=(1)$. If $n=6$, then $a=(b^{-1})^{n}ab^{n}$ for any $a,b \in S_3$. But $S_3$ is nonabelian.
\end{answer}

$$ $$

\begin{ex}
    If $G$ is a group, $a,b\in G$ and $bab^{-1}=a^{r}$ for some $r\in \mathbf{N}$, then $b^{j}ab^{-j}=a^{r^{j}}$ for all $j\in \mathbf{N}$. 
\end{ex}

\begin{answer}
    $bab^{-1}= a^{r}$. We prove it by induction. For $j=1$, its always true. Assume $j=k$ the equation is correct, $b^{k}ab^{-k}=a^{r^{k}}$. $ba^{r^{k}}b^{-1}=(a^{r^{k}})^{r=a^{r^{k+1}}}$. For $j=k+1$, it's also true.
\end{answer}

$$ $$

\begin{ex}
    If $a^{2}=e$ for all elements $a$ of a group $G$, then $G$ is abelian.
\end{ex}

\begin{answer}
    \[a^{2}=e\Rightarrow a^{2}a^{-1}=ea^{-1}=a(aa^{-1})=ae\Rightarrow a=a^{-1}\]\[ab=a^{-1}b^{-1}=(ab)^{-1}=(ba)^{-1}\]So  $ab=ba \forall a,b \in G$. $G$ is abelian.
\end{answer}

$$ $$

\begin{ex}
    If $G$ is a finite group of even order, then $G$ contains an element $a\neq e$ such that $a^{2}=e$.
\end{ex}

\begin{answer}
    Suppose not. $\forall a\neq e, aa\neq e \Leftrightarrow a\neq a^{-1}$. We can classify the group into some subsets. $G=\bigcup\limits_{a\neq e}\{a,a^{-1}\}\cup\{e\}$. Notice that $\{a,a^{-1}\}\cap\{b,b^{-1}\}=\emptyset$ if $a\neq b$, so $\left| G \right| =2n+1$, That's contradictory!
\end{answer}

$$ $$

\begin{ex}
    Let $G$ be a nonempty finite set with an associative binary operation such that for all $a, b, c\in G$, $ab = ac\Rightarrow b = c$ and $ba  =ca \Rightarrow b = c$. Then $G$ is a group. Show that this conclusion may be false if $G$ is finite.
\end{ex}

\begin{answer}
    $G$ is a semigroup. Fix $a\in G$ and take $b$ travels through all elements in $G$, then $ab$ travels through all elements in $G$.

    There exists an element $e_1$ s.t. $ae_1=a\forall a\in G$. Similarly, we can find $e_2$ s.t. $e_2a=a\forall a\in G$. $e_2e_1=e_1=e_2=e$. $e$ is the identity element of $G$. Easily, we can find that $\forall a\in G, \exists ! a^{-1}\in G$ s.t. $a^{-1}a=aa^{-1}=e$ because $ab = ac\Rightarrow b = c$ and $ba  =ca \Rightarrow b = c$.

    $G$ is a group. If $G$ is infinite, $G$ may not be a group, for example: $(Z_+,\times)$.
\end{answer}

$$ $$

\begin{ex}
    Let $a_1, a_2,\dots$ be a sequence of elements in a semigroup $G$. Then there exists a unique function $\Psi: \mathbf{N^*}\rightarrow G$ such that $\Psi(1)=a_1, \Psi(2)=a_1a_2, \Psi(3)=(a_1a_2)a_3$ and for $n\geq 1$, $\Psi(n+1)=(\Psi(n))a_{n+1}$. Note that $\Psi(n)$ is precisely the standard $n$ product $\prod_{i=1}^na_i$.
\end{ex}

\begin{answer}
    Applying the Recursion Theorem with $a=a_1, S=G$ and $f_n:G\to G$ given by $x\mapsto xa_{n+2}$ yields a function $\phi: \mathbf{N}\to G$. Let $\Psi=\phi\theta$, where $\theta:\mathbf{N^*}\to \mathbf{N}$ is given by $k\mapsto k-1$.
\end{answer}