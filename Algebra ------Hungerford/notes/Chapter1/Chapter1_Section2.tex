\section{Homomorphisms and subgroups}
\begin{definition}
    Let $G$ and $H$ be semigroups. A function $f : G \to H$ is a homomorphism provided
    \[f(ab) = f(a)f(b) \text{ for all } a,b \in G.\]
    If $f$ is injective as a map of sets, $f$ is said to be a monomorphism. If $f$ is surjectice, $f$ is called an epimorphism. If $f$ is bijective, $f$ is called an isomorphism. In this case $G$ and $H$ are said to be isomorphic (written $G \cong H$). A homomorphism $f : G \to G$ is called an endomorphism ofG and an isomorphism $f : G \to G$ is called an automorphism of $G$.
\end{definition}

\begin{definition}
    Let $f : G \to H$ be a homomorphism of groups. The kernel of $f$ (denoted $\mathrm{Ker}f$) is $\{ a \in G | f(a) = e \in H\}$ . If $A$ is a subset of $G$, then $f(A) = { b \in H | b = f(a) \text{ for some } a \in A}$ is the image of $A$. $f(G)$ is called the image of $f$ and denoted $\mathrm{Im} f$. If $B$ is a subset of $H$, $f^{-1}(B) =\{ a \in G | f(a) \in B \}$ is the inverse image of $B$.
\end{definition}