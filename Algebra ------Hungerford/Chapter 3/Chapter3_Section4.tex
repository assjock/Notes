\section{Rings of quotients and localization}
\begin{ex}
    Determine the complete ring of quotients of the ring $Z_{n}$ for each $n\geq 2$.
\end{ex}

$$ $$

\begin{ex}
    Let $S$ be a multiplicative subset of a commutative ring $R$ with identity and let $T$ be a multiplicative subset of the ring $S^{-1}R$. Let $S_{*}=\{r\in R|r /s\in T \text{ for some }s\in S\}$. Then $S_{*}$ is a multiplicative subset of $R$ and there is a ring isomorphism $S_{*}^{-1}R\cong T^{-1}(S^{-1}R)$.
\end{ex}

$$ $$

\begin{ex}
    \begin{enumerate}[(a)]
        \item The set $E$ of positive even integers is a multiplicative subset of $\mathbf{Z}$ such that $E^{-1}(\mathbf{Z})$ is field of rational numbers.
        \item State and prove condition(s) on a multiplicative subset of $S$ of $\mathbf{Z}$ which insure that $S^{-1}\mathbf{Z}$ is a field of rationals.
    \end{enumerate}
\end{ex}

$$ $$

\begin{ex}
    If $S=\{2,4\}$ and $R=Z_{6}$, then $S^{-1}R$ is isomorphic to the field $Z_{3}$. Consequently, the converse of Theorem 4.3(ii) is false.
\end{ex}

$$ $$

\begin{ex}
    Let $R$ be an integral domain with quotient field $F$. If $T$ is an integral domain such that $R\subset T\subset F$, then $F$ is (isomorphic to) the quotient field of $T$.
\end{ex}

$$ $$

\begin{ex}
    Let $S$ be a multiplicative subset of an integral domain $R$ such that $0\notin S$. If $R$ is a principle ideal domain, then so is $S^{-1}R$.
\end{ex}

$$ $$

\begin{ex}
    Let $R_{1}$ and $R_{2}$ be integral domains with quotient fields $F_{1}$ and $F_{2}$ respectively. If $f:R_{1}\to R_{2}$ is an isomorphism, then $f$ extends to an isomorphism $F_{1}\cong F_{2}$.
\end{ex}

$$ $$

\begin{ex}
    Let $R$ be a commutative ring with identity, $I$ an ideal of $R$ and $\pi:R\to R /I$ the canonical projection.
    \begin{enumerate}[(a)]
        \item If $S$ is a multiplicative subset of $R$, then $\pi S=\pi(S)$ is a multiplicative subset of $R /I$.
        \item The mapping $\theta:S^{-1}R\to (\pi S)^{-1}(R/I)$ given by $r /s\mapsto \pi(r) /\pi(s)$ is a well-defined function.
        \item $\theta$ is a ring epimorphism with kernel $S^{-1}I$ and hence induces a ring isomorphism $S^{-1}R /S^{-1}I\cong (\pi S)^{-1}(R /I)$.
    \end{enumerate}
\end{ex}

$$ $$

\begin{ex}
    Let $S$ be a multiplicative subset of a commutative ring $R$ with identity. If $I$ is an ideal in $R$, then $S^{-1}(\mathrm{Rad}I)=\mathrm{Rad}(S^{-1}I)$.
\end{ex}

$$ $$

\begin{ex}
    Let $R$ be an integral domain and for each maximal ideal $M$, consider $R_{M}$ as a subring of the quotient field of $R$. Show that $\cap R_{M}=R$, where the intersection is taken over all maximal ideals $M$ of $R$.
\end{ex}

$$ $$

\begin{ex}
    Let $p$ be a prime in $\mathbf{Z}$l then $(p)$ is a prime ideal. What can be said about the relationship of $Z_{p}$ and the localization $Z_{(p)}$?
\end{ex}

$$ $$

\begin{ex}
    A commutative ring with identity is local if and only if for all $r,s\in R$, $r+s=1_{R}$ implies $r$ or $s$ is a unit.
\end{ex}

$$ $$

\begin{ex}
    The ring $R$ consisting of all rational numbers with denominators not divisible by some (fixed) prime $p$ is a local ring.
\end{ex}

$$ $$

\begin{ex}
    If $M$ is a maximal ideal in a commutative ring $R$ with identity and $n$ is a positive integer, then the ring $R /M^{n}$ has a unique prime ideal and therefore is local.
\end{ex}

$$ $$

\begin{ex}
    In a commutative ring $R$ with identity the following conditionns are equivalent: (i) $R$ has a unique prime ideal; (ii) every nonunit is nilpotent; (iii) $R$ has a minimal prime ideal which contains all zero divisors, and all nonunits of $R$ are zero divisors.
\end{ex}

$$ $$

\begin{ex}
    Every nonzero homomorphic image of a local ring is local.
\end{ex}