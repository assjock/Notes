\section{Sets and their Operations}

$A$ \textbf{set} is a collection of well-defined objects.

If $A$ is a set, we write $a \in A$ to express element $a$ belongs to set $A$, the negetive proposition of which is $a \notin A$. We use the symbol $\oslash$ to denote the \textbf{empty set}, that is, the set with no elements.

\begin{theorem}[Cantor]\label{thm:Russell's paradox}
    There is no set contains all the sets.
\end{theorem}
\begin{proof}
    We assume $P(M)$ represents $M$ doesn't contain itself.

    Consider $K=\{M|P(M)\}$ which is made of sets $M$ that satisfies  $P$. Assuming $K$ is a set, then either $P(K)$ or $\lnot P(K)$ is true.
    
    If $P(K)$ is true, $K$ doesn't contain itself,but because of the definition of $K$, $K$ is belong to $K$, which means $\lnot P(M)$.

    If $\lnot P(M)$ is ture, it's easy to find the similar conclusion.

    So to the contrary, $K$ is not a set. This reveals a set can't contain all the sets.
\end{proof}

\thmref{thm:Russell's paradox} is a typical \textbf{paradox} called Russell's paradox.

$\forall $and $\exists $ are logical symbols to describe
\begin{table}[H]
    \centering
    \caption{Universial and Exsitential Quantifications}
    \begin{tabular}{|c|c|}\hline
        Symbols&Meanings\\\hline
        $\forall x \in A$&For all elements $x$ in $A$\\
        $\exists x \in A$&There exist at least one element $x$ in $A$\\
        \hline
    \end{tabular}
\end{table}
To show the inclusion relation of two sets, we often use the Symbol $A \subset B$, which means set $A$ is a \textbf{subset} of set $B$ (All the elements in $A$ also belong to $B$).We indicate that $A$ is not a subset of $B$ by this notaion: $A \not\subset B$.
\[(A\subset B ):=\forall x((x\in A)\Rightarrow (x\in B))\]

We define the equal relation between two sets, using the symbol $=$:
\[A=B :=(A\subset B)\wedge(B\subset A)\]
We often use this definition to prove $A=B$.
Symbol $\neq$ denotes the negetive proposition of equal.

$A$ is a \textbf{proper subset} of $B$, if $A$ is a subset of $B$, and $A\neq B$, denoted by the symbol $\subseteq $.

If $A$ and $B$ are sets, then their \textbf{union}, denoted by $A \cup  B$,is the set of all elements that are elements of either $A$ or $B$:
\[(A\cup B):=\{x\in M|(x\in A)\vee (x\in B)\}\]
Clearly, $A\cup B=B\cup A$.
\begin{figure}
    \centering
    \caption{Union of two sets}
    \input{img/Chapter1_Section2/Chapter1_Section2_img1.pdf_tex}
\end{figure}

If $A$ and $B$ are sets, then their \textbf{intersection}, denoted by $A \cap  B$,contains all the elements in both $A$ and $B$:
\[(A\cap B):=\{x\in M|(x\in A)\wedge (x\in B)\}\]
Also, we have $A\cap B=B\cap A$.
\begin{figure}[h] 
    \centering
    \caption{Intersection of two sets}
    \input{img/Chapter1_Section2/Chapter1_Section2_img2.pdf_tex}
\end{figure}

We use the denotion $A\backslash B$ to represent the set contains all the elements which belongs to $A$ but not belong to $B$, we call it the \textbf{defference set}.
\[A\backslash B:=\{x\in M|(x\in A)\wedge(x\not\in B)\}\]
\begin{figure}[ht]
    \centering
    \caption{Complement}
    \input{img/Chapter1_Section2/Chapter1_Section2_img3.pdf_tex}
\end{figure}

For $B\subset A$, we can also denote it as the symbol $C_{A}B$.

\begin{question}[de Morgen]
    \[C_{M}(A\cup B)=C_{M}A\cap C_{M}B\]
    \[C_{M}(A\cap B)=C_{M}A\cup C_{M}B\]
\end{question}
\begin{proof}We prove the first one.
    \[
    \begin{aligned}
        &(x\in C_{M}(A\cup B))\Rightarrow (x \not\in (A\cup B))\\ \Rightarrow& ((x\not\in A)\wedge(x\not\in B))
        \Rightarrow((x\in C_{m}A)\wedge(x\in C_{M}B))\\ \Rightarrow&((x\in C_{m}A)\cap(x\in C_{M}B))
    \end{aligned}\]
    So we have proved $C_{M}(A\cup B)\subset C_{M}A\cap C_{M}B$. On the other hand:
    \[\begin{aligned}
        &((x\in C_{m}A)\cap(x\in C_{M}B))\Rightarrow ((x\in C_{m}A)\wedge(x\in C_{M}B))\\ \Rightarrow& ((x\not\in A)\wedge(x\not\in B))\Rightarrow (x \not\in (A\cup B))\\ \Rightarrow&(x\in C_{M}(A\cup B))
    \end{aligned}\]
    That's the same as $C_{M}(A\cup B)=C_{M}A\cap C_{M}B$.
\end{proof}

\begin{question}
    $(A\subset C_{M}B)\Leftrightarrow (B\subset C_{M}A)$.
\end{question}
\begin{proof}
    \[\begin{aligned}
        &(A\subset C_{M}B)\Rightarrow ((x\in A)\Rightarrow((x\not\in B))\wedge(x\in M))\\ \Rightarrow& (\lnot(x\in A)\Leftarrow \lnot((x\not\in B))))\\
        \Rightarrow&((x\in B)\Rightarrow(x\not\in A))\Rightarrow(B\subset C_{M}A)
    \end{aligned}\]
    The other hand of this problem is the same.
\end{proof}
\begin{tips}
    \begin{enumerate}
        \item $(A\subset C)\wedge(B\subset C)\Leftrightarrow((A\cup B)\subset C)$;
        \item $(C\subset A)\wedge(C\subset B)\Leftrightarrow (C\subset(A\cap B))$;
        \item $C_{M}(C_{M}A)=A$;
        \item $(A\subset C_{M}B)\Leftrightarrow (B\subset C_{M}A)$;
        \item $(A\subset B)\Leftrightarrow(C_{M}A\supset C_{M}B)$.
    \end{enumerate}
\end{tips}
\begin{question}
    $A\cup (B\cap C)=(A\cup B)\cap (A\cup C)$.
\end{question}
\begin{proof}
    \[
        A\cup (B\cap C)\Leftrightarrow((x\in A)\vee ((x\in B))\wedge(x\in C))
    \]
    \[((x\in A)\vee(x\in B))\wedge((x\in A)\vee(x\in C))\Leftrightarrow(A\cup B)\cap (A\cup C)\]
    So, we should prove:
    \[((x\in A)\vee ((x\in B))\wedge(x\in C))\Leftrightarrow((x\in A)\vee(x\in B))\wedge((x\in A)\vee(x\in C))\]
    That's the same as:
    \[(A\vee(B\wedge C))\Leftrightarrow((A\vee B)\wedge(A\vee C))\]
    It's easy to prove with the help of truth table.
\end{proof}
In this question, we also establish a relation between logical operation and sets' operation.
\begin{tips}
    \begin{enumerate}
        \item $A\cup (B\cup C)=(A\cup B)\cup C:=A\cup B\cup C$;
        \item $A\cap (B\cap C)=(A\cap B)\cap C:=A\cap B\cap C$;
        \item $A\cup (B\cap C)=(A\cup B)\cap (A\cup C)$;
        \item $A\cap (B\cup C)=(A\cap B)\cup (A\cap C)$.
    \end{enumerate}
\end{tips}
We denote two sets' \textbf{Cartesian product} as $A \times B $.
$A\times B$ is a set contains ordered pairs, which means $A\times B\neq B\times A$.
\[A\times B:=\{(x,y)|(x\in A)\wedge(y\in B)\}\]
\begin{question}
    $(X\times Y)\cup (Z\times Y)=(X\cup Z)\times Y$.
\end{question}
\begin{proof}
    \[\begin{aligned}
        &(X\times Y)\cup (Z\times Y)\\\Rightarrow&\{(x,y)|(x\in X)\wedge(y\in Y)\}\cup \{(z,y)|(z\in Z)\wedge(y\in Y)\}\\\Rightarrow&\{(x ({\rm or}\ z),y)|((x\in X)\wedge(y\in Y))\vee((z\in Z)\wedge(y\in Y))\}\\\Rightarrow&\{(x ({\rm or}\ z),y)|((x\in X)\vee (z\in Z))\wedge(y\in Y)\}\\\Rightarrow&(X\cup Z)\times Y
    \end{aligned}\]
    On the other hand:
    \[\begin{aligned}
        &(X\cup Z)\times Y\\\Rightarrow&\{(x ({\rm or}\ z),y)|((x\in X)\vee (z\in Z))\wedge(y\in Y)\}\\\Rightarrow&\{(x ({\rm or}\ z),y)|((x\in X)\wedge(y\in Y))\vee((z\in Z)\wedge(y\in Y))\}\\\Rightarrow&\{(x,y)|(x\in X)\wedge(y\in Y)\}\cup \{(z,y)|(z\in Z)\wedge(y\in Y)\}\\\Rightarrow&(X\times Y)\cup (Z\times Y)
    \end{aligned}\]
    In conclusion:$(X\times Y)\cup (Z\times Y)=(X\cup Z)\times Y$.
\end{proof}
\begin{question}
    $(A\times B\subset X\times Y)\Leftrightarrow(A\subset X)\wedge(B\subset Y)$.
\end{question}
\begin{proof}
    \[\begin{aligned}
        (A\times B)\subset (X\times Y)\Rightarrow(((a\in A)\wedge(b\in B))\Rightarrow((x\in X)\wedge(y\in Y)))
    \end{aligned}\]
    In this formula, we will get \[((a\in A)\Rightarrow(x\in X))\wedge((b\in B)\Rightarrow(y\in Y))\] That's because $A\times B$ is a ordered pair, which means there is a consistent one-to-one match between $A$ and $X$, $B$ and $Y$.
    
    The proof of the other hand is similar.
\end{proof}
\begin{tips}
    \begin{enumerate}
        \item $(X\times Y=\oslash)\Leftrightarrow(X=\oslash)\vee(Y=\oslash)$,\\{\rm if} $X\times Y\neq \oslash, A\times B\neq \oslash$, {\rm we have}
        \item $(A\times B\subset X\times Y)\Leftrightarrow(A\subset X)\wedge(B\subset Y)$;
        \item $(X\times Y)\cup (Z\times Y)=(X\cup Z)\times Y$
        \item $(X\times Y)\cap (X'\times Y')=(X\cap X')\times (Y\times Y')$.
    \end{enumerate}
\end{tips}