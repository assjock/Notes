\section{Sets and their Operations}

$A$ set is a collection of well-defined objects.

If $A$ is a set, we write $a \in A$ to express element $a$ belongs to set $A$, the negetive proposition of which is $a \notin A$. We use the symbol $\oslash$ to denote the empty set, that is, the set with no elements.

\begin{theorem}[Cantor]\label{thm:Russell's paradox}
    There is no set contains all the sets.
\end{theorem}
\begin{proof}
    We assume $P(M)$ represents $M$ doesn't contain itself.

    Consider $K=\{M|P(M)\}$ which is made of \textbf{sets} $M$ that satisfies  $P$. Assuming $K$ is a set, then either $P(K)$ or $\lnot P(K)$ is true.
    
    If $P(K)$ is true, $K$ doesn't contain itself,but because of the definition of $K$, $K$ is belong to $K$, which means $\lnot P(M)$.

    If $\lnot P(M)$ is ture, it's easy to find the similar conclusion.

    So to the contrary, $K$ is not a set. This reveals a set can't contain all the sets.
\end{proof}

\thmref{thm:Russell's paradox} is a typical paradox called Russell's paradox.

$\forall $and $\exists $ are logical symbols to describe
\begin{table}[H]
    \centering
    \caption{Universial and Exsitential Quantifications}
    \begin{tabular}{|c|c|}\hline
        Symbols&Meanings\\\hline
        $\forall x \in A$&For all elements $x$ in $A$\\
        $\exists x \in A$&There exist at least one element $x$ in $A$\\
        \hline
    \end{tabular}
\end{table}
To show the inclusion relation of two sets, we often use the Symbol $A \subset B$, which means set $A$ is a \textbf{subset} of set $B$ (All the elements in $A$ also belong to $B$).We indicate that $A$ is not a subset of $B$ by this notaion: $A \not\subset B$.
\[(A\subset B ):=\forall x((x\in A)\Rightarrow (x\in B))\]

We define the equal relation between two sets, using the symbol $=$:
\[A=B =:(A\subset B)\wedge(B\subset A)\]
We often use this definition to prove $A=B$.
Symbol $\neq$ denotes the negetive proposition of equal.

$A$ is a \textbf{proper subset} of $B$, if $A$ is a subset of $B$, and $A\neq B$, denoted by the symbol $\subseteq $.

If $A$ and $B$ are sets, then their \textbf{union}, denoted by $A \cup  B$,is the set of all elements that are elements of either $A$ or $B$:
\[(A\cup B)=:\{x\in M|(x\in A)\vee (x\in B)\}\]
Clearly, $A\cup B=B\cup A$.
\begin{figure}
    \centering
    \caption{Union of two sets}
    \input{img/img2.pdf_tex}
\end{figure}

If $A$ and $B$ are sets, then their \textbf{intersection}, denoted by $A \cap  B$,contains all the elements in both $A$ and $B$:
\[(A\cap B)=:\{x\in M|(x\in A)\wedge (x\in B)\}\]
Also, we have $A\cap B=B\cap A$.
\begin{figure}[h] 
    \centering
    \caption{Intersection of two sets}
    \input{img/img1.pdf_tex}
\end{figure}

We use the donation $A\backslash B$ to represent the set contains all the elements which belongs to $A$ but not belong to $B$.
\[A\backslash B=:\{x\in M|(x\in A)\wedge(x\not\in B)\}\]
\begin{figure}[h]
    \centering
    \caption{Complement}
    \input{img/img3.pdf_tex}
\end{figure}

For $B\subset A$, we can also denote it as the symbol $C_{A}B$.

\begin{question}[de Morgen]
    \[C_{M}(A\cup B)=C_{M}A\cap C_{M}B\]
    \[C_{M}(A\cap B)=C_{M}A\cup C_{M}B\]
\end{question}
\begin{proof}We prove the first one.
    \[
    \begin{aligned}
        &(x\in C_{M}(A\cup B))\Rightarrow (x \not\in (A\cup B))\\ &\Rightarrow ((x\not\in A)\wedge(x\not\in B))
        \Rightarrow((x\in C_{m}A)\wedge(x\in C_{M}B))\\ &\Rightarrow((x\in C_{m}A)\cap(x\in C_{M}B))
    \end{aligned}\]
    So we have proved $C_{M}(A\cup B)\subset C_{M}A\cap C_{M}B$. On the other hand:
    \[\begin{aligned}
        &((x\in C_{m}A)\cap(x\in C_{M}B))\Rightarrow ((x\in C_{m}A)\wedge(x\in C_{M}B))\\ &\Rightarrow ((x\not\in A)\wedge(x\not\in B))\Rightarrow (x \not\in (A\cup B))\\ &\Rightarrow(x\in C_{M}(A\cup B))
    \end{aligned}\]
    That's the same as $C_{M}(A\cup B)=C_{M}A\cap C_{M}B$.
\end{proof}
